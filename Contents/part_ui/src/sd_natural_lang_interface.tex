\begin{SCn}
    \scnsectionheader{Предметная область и онтология естественно-языковых интерфейсов ostis-систем}
    \begin{scnsubstruct}
\begin{scnrelfromlist}{соавтор}
		\scnitem{Никифоров С.А.}
        \scnitem{Гойло А.А.}
        \scnitem{Цянь Л.}
	\end{scnrelfromlist}
    \begin{scnrelfromlist}{дочерний раздел}
        \scnitem{Предметная область и онтология синтаксического анализа естественно-языковых сообщений, входящих в ostis-систему}
        \scnitem{Предметная область и онтология понимания естественно-языковых сообщений, входящих в ostis-систему}
        \scnitem{Предметная область и онтология синтеза естественно-языковых сообщений  ostis-системы}
    \end{scnrelfromlist}
    
    \begin{scnrelfromlist}{ключевой знак}
    	\scnitem{Абстрактный sc-агент лексического анализа}
    	\scnitem{Абстрактный sc-агент синтаксического анализа}
    	\scnitem{Абстрактный sc-агент понимания сообщения}
    \end{scnrelfromlist}

    \begin{scnrelfromlist}{ключевое понятие}
        \scnitem{естественно-языковой интерфейс}
        \scnitem{речевой интерфейс}
        \scnitem{действие. лексический анализ естественно-языкового сообщения}
        \scnitem{действие. синтаксический анализ естественно-языкового сообщения}
        \scnitem{действие. понимание естественно-языкового сообщения}
        \scnitem{контекст}
        \scnitem{контекст диалога}
    \end{scnrelfromlist}

    \begin{scnrelfromlist}{ключевое отношение}
        \scnitem{потенциально эквивалентная структура*}
        \scnitem{множество тематических контекстов диалога*}
    \end{scnrelfromlist}

    \begin{scnrelfromlist}{ключевое знание}
        \scnitem{Структура решателя задач естественно-языкового интерфейса ostis-систем}
        \scnitem{Синтаксический анализ естественно-языкового сообщения}
        \scnitem{Понимание естественно-языкового сообщения}
        \scnitem{Разрешение контекста}
        \scnitem{Принципы построения естественно-языкового интерфейса ostis-систем для китайского языка}
    \end{scnrelfromlist}

    \begin{scnrelfromlist}{библиографическая ссылка}
		\scnitem{\scncite{GlobalMarket}}
		\scnitem{\scncite{Pais2022}}
		\scnitem{\scncite{Trajanov2022}}
		\scnitem{\scncite{Khurana2022}}
		\scnitem{\scncite{Strubell2019}}
		\scnitem{\scncite{LLM}}
		\scnitem{\scncite{Alexa}}
		\scnitem{\scncite{Siri}}
		\scnitem{\scncite{GoogleAssistant}}
		\scnitem{\scncite{Cortana}}
		\scnitem{\scncite{Hoy2018}}
		\scnitem{\scncite{Jackendoff1977}}
		\scnitem{\scncite{Davydenko2018}}
		\scnitem{\scncite{Shunkevich2018}}
		\scnitem{\scncite{Davydenko2017}}
	\end{scnrelfromlist}

    \scntext{аннотация}{В данной предметной области рассматривается подход к реализации \textit{естественно-языковых интерфейсов} \textit{ostis-систем}, построенных по \textit{Технологии OSTIS}, а также предлагается модель \textit{контекста диалога}. В данном подходе все этапы анализа, включая \textit{лексический}, \textit{синтаксический} и \textit{семантический анализ} могут производиться непосредственно в \textit{базе знаний} такой системы. Такой подход позволит эффективно решать такие задачи как управление глобальным и локальным \textit{контекстами диалога}, а также разрешение языковых явлений таких как анафоры, омонимия и эллиптические фразы.}
    \scntext{цель}{Формирование модели \textit{интерфейса}, в основе которой лежит подход к обработке \textit{естественного языка} на основе \textit{онтологий}, содержащих формальное описание \textit{естественного языка}.}

    \scnheader{естественно-языковой интерфейс ostis-систем}
    \scntext{примечание}{Одной из основных особенностей \textit{ostis-систем} должен являться \textit{пользовательский интерфейс}, способный обеспечить эффективное взаимодействие пользователя с системой в условиях его общей профессиональной неподготовленности.}

    \scnheader{естественно-языковой интерфейс}
    \scnsuperset{речевой интерфейс}
    \begin{scnindent}
        \scnrelfrom{источник}{Предметная область и онтология интерфейсов ostis-систем}
    \end{scnindent}
    \scnidtf{SILK-интерфейс}
    \begin{scnrelfromlist}{примечание}
        \scnfileitem{В настоящее время существует большое количество различных \textit{интерфейсов} компьютерных систем, что усложняет интероперабельность между такими системами и людьми в силу необходимости ознакомления с интерфейсом каждой новой системы, который зачастую может быть не интуитивно понятен.}
        \scnfileitem{Одной из наиболее естественных и удобных форм передачи информации между людьми является речь, что обуславливает все большее распространение \textit{естественно-языковых интерфейсов}. В настоящий момент времени уже ни у кого не вызывает сомнения, что данная форма взаимодействия человека и машины играет и будет играть значительную роль во взаимодействии с различными компьютерными системами.}
        \begin{scnindent}
            \scnrelfrom{источник}{\scncite{GlobalMarket}}
        \end{scnindent}
        \scnfileitem{Необходимо отметить, что большое многообразие \textit{языков} (как \textit{естественных}, так и \textit{искусственных}) ведет к необходимости упрощения процесса создания таких \textit{интерфейсов} для каждого отдельно взятого \textit{языка}.}
    \end{scnrelfromlist}

    \scnheader{разработка естественно-языковых интерфейсов для современных интеллектуальных систем, основанных на знаниях}
    \begin{scnrelfromlist}{основные аспекты}
        \scnitem {особенности обрабатываемого естественного языка}
        \scnitem {диапазон базы знаний интеллектуальных систем, то есть широта знаний в базах знаний интеллектуальных систем}
    \end{scnrelfromlist}

    \scnheader{методы обработки естественного языка}
    \begin{scnrelfromlist}{основные направления}
        \scnitem {методы на основе правил и лингвистических знаний}
        \scnitem {методы машинного обучения, основанные на математической статистике и теории информации}
    \end{scnrelfromlist}
    \begin{scnrelfromlist}{примечание}
        \scnfileitem{В основе большинства подходов к обработке и пониманию \textit{естественного языка} лежит машинное обучение. Несомненно, для большинства широко распространенных \textit{языков} модели для обработки естественного языка работают очень хорошо и совершенствуются с каждым днем, но несмотря на успехи в данной области, данный подход имеет ряд недостатков:
            \begin{itemize}
                \item{Проблемы при работе с различными областями, например, значения слов или предложений могут быть различными в зависимости от \textit{предметной области}. Таким образом, модели для NLP могут хорошо работать для отдельной \textit{предметной области}, но не подходить для широкого применения.}
                \item{Создание новой модели модели требует наличия большого объема данных, а качество таких данных напрямую влияет на качество получаемой модели, что ведет к большим затратам на ее обучение.}
                \item{Данные модели представляют собой \scnqq{черный ящик}, так как данные модели не обладают средствами для обоснования своего вывода.}
                \item{Каждая такая модель решает только свой узкий класс задач, отсутствует общий подход к обработке естественного языка.}
            \end{itemize}}
            \begin{scnindent}
                \begin{scnrelfromlist}{источник}
                \scnitem{\scncite{Pais2022}}
                \scnitem{\scncite{Trajanov2022}}
                \scnitem{\scncite{Khurana2022}}
                \scnitem{\scncite{Strubell2019}}
                \scnitem{\scncite{LLM}}
                \scnitem{\scncite{Khurana2022}}
                \end{scnrelfromlist}
            \end{scnindent}
        \scnfileitem{Недостатки используемых методов обработки естественного языка являются причиной части недостатков современных систем, реализующих \textit{естественно-языковой интерфейс}, так, несмотря на то, что сейчас существует большое количество речевых ассистентов, создаваемых разными компаниями. Они обладают схожими недостатками, например, исключительно распределенной реализацией, в силу недостаточной для запуска ресурсоемких моделей производительности устройств конечных пользователей. Это в свою очередь ведет к проблемам с приватностью.}
        \begin{scnindent}
            \begin{scnrelfromlist}{источник}
                \scnitem{\scncite{Alexa}}
                \scnitem{\scncite{Siri}}
                \scnitem{\scncite{GoogleAssistant}}
                \scnitem{\scncite{Cortana}}
                \scnitem{\scncite{Hoy2018}}
            \end{scnrelfromlist}
        \end{scnindent}
        \scnfileitem{Подмодуль понимания речи данных систем формирует конструкцию, отражающую смысл сообщения используя \textit{фреймовую модель}.}
        \begin{scnindent}
            \scnrelfrom{упрощенный пример}{\scnfileimage[40em]{Contents/part_ui/src/images/sd_ui/message_intents.png}}
            \begin{scnindent}
                \scnidtf{Рисунок. Иллюстрация формализованного смысла сообщения}
            \end{scnindent}
        \end{scnindent}
        \scnfileitem{Для представления результатов промежуточных этапов обработки используются иные форматы, модули которые их реализуют не имеют какой-либо единой основы и взаимодействуют посредством специализированных \textit{программных интерфейсов} между ними, что приводит к несовместимости способов представления результатов на различных этапах обработки и конечного результата обработки текстов. Данная несовместимость в свою очередь ведет к существенным накладным расходам при разработке такой системы и в особенности при ее модификации.}
        \scnfileitem{В качестве решения проблемы совместимости предлагается использование подхода к обработке \textit{естественного языка} на основе его \textit{формальной модели} в виде набора \textit{онтологий}, сформированных с использованием универсальных средств представления знаний, что будет способствовать интероперабельности как компонента по обработке \textit{естественного языка} в целом с другими компонентами системы, так и между составляющими самого данного компонента.}
    \end{scnrelfromlist}
        
    \scnheader{обработка естественного языка на основе его формальной модели}
    \begin{scnrelfromvector}{этапы обработки естественного языка}
        \scnitem{лексический анализ}
        \begin{scnindent}
            \begin{scnrelfromvector}{включение}
                \scnitem{декомпозиция текста на токены}
                \scnitem{сопоставление токенов с лексемами}
            \end{scnrelfromvector}
        \end{scnindent}
        \scnitem{синтаксический анализ}
        \scnitem{понимание сообщения}
        \begin{scnindent}
            \begin{scnrelfromvector}{включение}
                \scnitem{генерация вариантов значения сообщения}
                \scnitem{выбор из всех вариантов значения сообщения корректного на основании контекста}
                \scnitem{погружение сообщения в контекст}
            \end{scnrelfromvector}
        \end{scnindent}
    \end{scnrelfromvector}
    \scntext{примечание}{При анализе сообщения на \textit{естественном языке} используются средства введенные в Предметной области информационных конструкций и языков.}
    \begin{scnindent}
        \scnrelfrom{источник}{Предметная область и онтология интерфейсов ostis-систем}
    \end{scnindent}

    \scnheader{Решатель задач естественно-языкового интерфейса}
    \begin{scnrelfromset}{декомпозиция абстрактного sc-агента}
        \scnitem{Абстрактный sc-агент лексического анализа}
        \begin{scnindent}
            \begin{scnrelfromset}{декомпозиция абстрактного sc-агента}
                \scnitem{Абстрактный sc-агент декомпозиции текста на токены}
                \scnitem{Абстрактный sc-агент сопоставления токенов с лексемами}
            \end{scnrelfromset}
        \end{scnindent}
        \scnitem{Абстрактный sc-агент синтаксического анализа}
        \scnitem{Абстрактный sc-агент понимания сообщения}
        \begin{scnindent}
            \begin{scnrelfromset}{декомпозиция абстрактного sc-агента}
                \scnitem{Абстрактный sc-агент генерации вариантов значения сообщения}
                \scnitem{Абстрактный sc-агент выбора и обновления контекста}
                \begin{scnindent}
                    \begin{scnrelfromset}{декомпозиция абстрактного sc-агента}
                        \scnitem{Абстрактный sc-агент разрешения контекста}
                        \scnitem{Абстрактный sc-агент выбора смысла сообщения на основе контекста}
                        \scnitem{Абстрактный sc-агент погружения сообщения в контекст}
                    \end{scnrelfromset}
                \end{scnindent}
            \end{scnrelfromset}
        \end{scnindent}
    \end{scnrelfromset}

    \scnheader{SCg-текст. Иллюстрация спецификации агента}
	\scnrelfrom{иллюстрация}{\scnfileimage[40em]{Contents/part_ui/src/images/sd_ui/agent_spec.png}}
    \begin{scnindent}
        \scnidtf{пример фрагмента спецификации агента}
        \scntext{примечание}{Для каждого \textit{агента} в базе знаний должна находиться спецификация.}
    \end{scnindent}

    \end{scnsubstruct}
\end{SCn}
    