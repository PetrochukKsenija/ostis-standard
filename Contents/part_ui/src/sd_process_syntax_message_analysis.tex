\begin{SCn}
    \scnsectionheader{Предметная область и онтология синтаксического анализа естественно-языковых сообщений, входящих в ostis-систему}
    \begin{scnsubstruct}
    \begin{scnrelfromlist}{соавтор}
        \scnitem{Никифоров С.А.}
        \scnitem{Гойло А.А.}
        \scnitem{Цянь Л.}
    \end{scnrelfromlist}

    \scnheader{Предметная область и онтология синтаксического анализа естественно-языковых сообщений, входящих в ostis-систему}
    \scntext{введение}{В данной предметной области не будет подробно описан процесс лексического анализа и его ключевые аспекты (такие как, например, устранение омонимии) --- данный вопрос требует дополнительной проработки. Вместо этого, акцент будет сделан на этап синтаксического анализа, предварительным условием которого является уже проведенный лексический анализ текста \textit{естественного языка}. Лексический анализ основывается на средствах введенных в Предметной области и онтологии синтаксиса естественных языков.}
    \begin{scnhaselementrolelist}{класс объектов исследования}
        \scnitem{лексический анализ}
        \scnitem{действие. лексический анализ естественно-языкового сообщения}
        \scnitem{лексема}
        \scnitem{единица сегментации}
        \scnitem{синтаксический анализ}
    \end{scnhaselementrolelist}

    \scnheader{действие. лексический анализ естественно-языкового сообщения}
    \begin{scnrelfromset}{обобщенная декомпозиция}
        \scnitem{действие. декомпозиция текста на токены}
        \scnitem{действие. сопоставление токенов с лексемами}
    \end{scnrelfromset}

    \scnheader{лексический анализ}
    \scntext{примечание}{С точки зрения \textit{ostis-системы}, любой \textit{естественно-языковой} текст является \textit{файлом}}
        \begin{scnindent}
            \scnrelfrom{источник}{Предметная область и онтология файлов, внешних информационных конструкций и внешних языков ostis-систем}
        \end{scnindent}
    \scntext{пояснение}{Лексический анализ представляет собой декомпозицию текста на последовательность токенов и сопоставление \textit{лексем} с получившимися при данной декомпозиции токенами. Следует отметить, что данные токены при необходимости могут сопоставляться не с \textit{лексемами}, а с их подмножествами, входящими в ее \textit{морфологическую парадигму}, соответствующими определенным грамматическим категориям: падежу, числу, роду и так далее.}
    \scnrelfrom{пример результата}{\scnfileimage[40em]{Contents/part_ui/src/images/sd_ui/lexical.png}}
    \begin{scnindent}
        \scnidtf{SCg-текст. Иллюстрация результата лексического анализа}
    \end{scnindent}
    \scntext{примечание}{Для осуществления лексического анализа, в базе знаний системы также должен присутствовать словарь, содержащий \textit{лексемы} и их различные формы.}
    
    \scnheader{лексема}
    \scntext{определение}{\textbf{\textit{лексема}} --- единица словарного состава языка, которая представляет собой множество всех форм некоторого слова}
    \scnrelfrom{пример спецификации}{SCg-текст. Иллюстрация к спецификации лексемы в базе знаний.}
    \begin{scnindent}
        \scnrelfrom{источник}{Предметная область и онтология естественных языков}
    \end{scnindent}

    \scnheader{единица сегментации}
	\scntext{определение}{базовая единица для обработки китайского языка, имеющая определенные семантические или грамматические свойства}
	\scnsubset{файл}
    \begin{scnrelfromlist}{примечание}
        \scnfileitem{В Предметной области и онтологии синтаксиса естественных языков предлагалась формализация лингвистических знаний, совместимых в первую очередь с европейскими языками. В данной предметной области мы также рассмотрим, как могут учитываться особенности конкретных естественных языков при разработке естественно-языковых интерфейсов ostis-систем на примере китайского языка.}
        \scnfileitem{Традиционно в лингвистике структура слова изучается в рамках морфологии. Носителем морфологической парадигмы и ключевым элементом анализа является лексема. Однако, в китайском языке из-за письменной традиции (текст китайского языка состоит из последовательности иероглифов без пробелов), наименьшей единицей при обработке текстов считается единица сегментации. Описание единицы сегмантации приводится в государственном стандарте «Стандарт сегментации слов современного китайского языка, используемый для обработки информации».}
        \scnfileitem{Стоит отметить, что термин единица сегментации используется при компьютерной обработке текстов китайского языка и не полностью совпадает с описанием слов в китайской лингвистике.}
        \scnfileitem{Кроме того, в китайском языке отсутствуют четкие показатели категорий числа, падежа и рода, в отличие от русского языка и других европейских языков. Функцию слова в китайском языке можно определить не на основании морфемного состава, а при помощи анализа связей этого слова с другими словами. В связи с этим, в процессе анализа текстов китайского языка сначала необходимо выполнить лексический анализ, разбивающий поток иероглифов в тексте китайского языка на отдельные значимые единицы сегментации.}
    \end{scnrelfromlist}
    \scnrelfrom{результат декомпозиции предложения на единицы сегментации}{\scnfileimage[40em]{Contents/part_ui/src/images/sd_ui/segment_chinese_sentence.png}}
    \begin{scnindent}
        \scnidtf{SCg-текст. Иллюстрация результата лексического анализа предложения на китайском языке}
    \end{scnindent}

    \scnheader{синтаксический анализ}
    \begin{scnrelfromlist}{примечание}
        \scnfileitem{Агент синтаксического анализа выполняет переход от размеченного на \textit{лексемы} текста к его \textit{синтаксической структуре}.}
        \begin{scnindent}
            \scnrelfrom{источник}{Предметная область и онтология информационных конструкций и языков}
        \end{scnindent}
        \scnfileitem{При этом из-за невозможности разрешения структурной неоднозначности на этапе синтаксического анализа, его результатом в общем случае будет являться множество потенциальных синтаксических структур.}
        \scnfileitem{Синтаксический анализ также основывается на средствах введенных в в Предметной области и онтологии синтаксиса естественных языков}
        \begin{scnindent}
            \begin{scnrelfromset}{пример синтексической структуры}
                \scnitem{SCg-текст. Иллюстрация синтаксической структуры предложения. Первая часть}
                \scnitem{SCg-текст. Иллюстрация синтаксической структуры предложения. Вторая часть}
            \end{scnrelfromset}
        \end{scnindent}
    \end{scnrelfromlist}

    \end{scnsubstruct}
\end{SCn}
    