\begin{SCn}
	\scnsectionheader{Сегмент. Понятие библиотеки многократно используемых компонентов ostis-систем}

\begin{scnsubstruct}
	
	 \scnheader{библиотека многократно используемых компонентов ostis-систем}
	 \scntext{часто используемый sc-идентификатор}{библиотека компонентов ostis-систем}
	 \scntext{часто используемый sc-идентификатор}{библиотека компонентов}
	 \scnidtf{библиотека совместимых многократно используемых компонентов}
	 \scnidtf{комплексная библиотека многократно используемых семантически совместимых компонентов ostis-систем}
	 \scnidtf{библиотека многократно используемых и совместимых компонентов интеллектуальных компьютерных систем нового поколения}
	 \scnidtf{библиотека типовых компонентов ostis-систем}
	 \scnidtf{библиотека многократно используемых компонентов OSTIS}
	 \scnidtf{библиотека повторно используемых компонентов OSTIS}
	 \scnidtf{библиотека intelligent property компонентов ostis-систем}
	 \begin{scnindent}
	 	\scntext{сокращение}{библиотека ip-компонентов ostis-систем}
	 \end{scnindent}
	 \scntext{примечание}{библиотека многократно используемых компонентов ostis-систем позволяет использовать проектный опыт по разработке и модернизации ostis-систем различного назначения.}
	 \scnhaselementrole{типичный пример}{\scnkeyword{Библиотека Метасистемы OSTIS}}
	 \begin{scnindent}
	 	\scnidtf{Распределенная библиотека типовых (многократно используемых) компонентов ostis-систем в составе Метасистемы OSTIS}
	 	\scnidtf{Библиотека многократно используемых компонентов ostis-систем в составе \textit{Метасистемы OSTIS}}
	 \end{scnindent}
	 \scnhaselementrole{типичный пример}{\scnkeyword{Библиотека Экосистемы OSTIS}}
	 \begin{scnindent}
	 	\scntext{часто используемый sc-идентификатор}{Библиотека OSTIS}
	 	\scnidtf{Библиотека многократно используемых и совместимых компонентов интеллектуальных компьютерных систем нового поколения}
	 	\scnidtf{Библиотека типовых компонентов интеллектуальных компьютерных систем нового поколения}
	 	\scnidtf{Распределенная библиотека типовых (многократно используемых) компонентов ostis-систем в составе Экосистемы OSTIS}
	 	\scnidtf{Библиотека многократно используемых компонентов ostis-систем в составе \textit{Экосистемы OSTIS}}
	 	\scntext{примечание}{Все библиотеки в рамках \textit{Экосистемы OSTIS} объединяются в \textit{Библиотеку Экосистемы OSTIS}.}
	 	\scntext{примечание}{Постоянно расширяемая Библиотека Экосистемы OSTIS существенно сокращает сроки разработки новых интеллектуальных компьютерных систем.}
	 	\scntext{назначение}{Основное назначение Библиотеки Экосистемы OSTIS --- создание условий для эффективного, осмысленного и массового проектирования ostis-систем и их компонентов путём создания среды для накопления и совместного использования компонентов ostis-систем.}
	 	\begin{scnindent}
	 		\scntext{примечание}{Такие условия осуществляются путём неограниченного расширения постоянно эволюционируемых семантически совместимых ostis-систем и их компонентов, входящих в \textit{Экосистему OSTIS}.}
	 	\end{scnindent}
	 	\scntext{примечание}{Различные \textit{многократно используемые компоненты ostis-систем} объединяются в \textit{библиотеки многократно используемых компонентов ostis-систем}. Разработчики \uline{любой} \textit{ostis-системы} могут включить в ее состав библиотеку, которая позволит им накапливать и распространять результаты своей деятельности среди других участников \textit{Экосистемы OSTIS} в виде \scnkeyword{многократно используемых компонентов}. Решение о включении компонента в библиотеку принимается экспертным сообществом разработчиков, ответственным за качество этой библиотеки. Верификацию компонентов можно автоматизировать путем проверки наличия обязательной части их спецификации, а также тестированием корректности автоматической установки, интеграции и функционирования компонентов.}
	 \end{scnindent}
	 \scntext{примечание}{В рамках \textit{Экосистемы OSTIS} существует множество библиотек многократно используемых компонентов ostis-систем, являющихся подсистемами соответствующих материнских ostis-систем. Главной библиотекой многократно используемых компонентов ostis-систем является \textit{Библиотека Метасистемы OSTIS}. \textit{Метасистема OSTIS} выступает \scnkeyword{материнской системой} для всех разрабатываемых ostis-систем, поскольку содержит все базовые компоненты.}
	 \begin{scnindent}
	 	\scnrelfrom{описание примера}{\scnfileimage[35em]{Contents/part_methods_tools/src/images/sd_ostis_library/ecosystem_architecture.png}}
	 	\begin{scnindent}
	 		\scnrelfrom{смотрите}{менеджер многократно используемых компонентов ostis-систем}
	 	\end{scnindent}
	 \end{scnindent}
	 \begin{scnrelfromset}{функциональные возможности}
	 	\scnfileitem{Хранение многократно используемых компонентов ostis-систем и их спецификаций.}
	 	\begin{scnindent}
	 		\scntext{примечание}{При этом часть компонентов, специфицированных в рамках библиотеки, могут физически храниться в другом месте ввиду особенностей их  технической реализации (например, исходные тексты платформы интерпретации sc-моделей компьютерных систем могут физически храниться в каком-либо отдельном репозитории, но специфицированы как компонент будут в соответствующей библиотеке). В этом случае спецификация компонента в рамках библиотеки должна также включать описание (1) того, где располагается компонент, и (2) сценария его автоматической или хотя бы ручной установки в дочернюю ostis-систему.}
	 		\begin{scnindent}
	 			\scnrelfrom{смотрите}{менеджер многократно используемых компонентов ostis-систем}
	 		\end{scnindent}
	 	\end{scnindent}
	 	\scnfileitem{Просмотр имеющихся компонентов и их спецификаций, а также поиск компонентов по фрагментам их спецификации.}
	 	\scnfileitem{Хранение сведений о том, в каких ostis-системах-потребителях какие из компонентов библиотеки и какой версии используются (были скачаны). Это необходимо как минимум для учета востребованности того или иного компонента, оценки его важности и популярности.}
	 	\scnfileitem{Систематизация многократно используемых компонентов ostis-систем.}
	 	\scnfileitem{Обеспечение версионирования многократно используемых компонентов ostis-систем.}
	 	\scnfileitem{Поиск зависимостей между многократно используемыми компонентами в рамках библиотеки компонентов.}
	 	\scnfileitem{Обеспечение автоматического обновления компонентов, заимствованных в дочерние ostis-системы. Данная функция может включаться и отключаться по желанию разработчиков дочерней ostis-системы.}
	 	\begin{scnindent}
	 		\scntext{примечание}{Одновременное обновление одних и тех же компонентов во всех системах, его использующих, не должно ни в каком контексте приводить к несогласованности между этими системами. Это требование может оказаться довольно сложным, но без него работа Экосистемы невозможна.}
	 	\end{scnindent}
	 \end{scnrelfromset}
	 \begin{scnindent}
	 	\scnrelfrom{источник}{\cite{Koronchik2011}}
	 \end{scnindent}
	 \scntext{примечание}{\scnkeyword{библиотека многократно используемых компонентов ostis-систем} позволяет избавиться от дублирования семантически эквивалентных информационных компонентов. А также от многообразия форм технической реализации используемых моделей решения задач.}
	 \scntext{примечание}{Проблема интеграции многократно используемых компонентов ostis-систем решается путем взаимодействия компонентов через общую базу знаний. Компоненты могут использоватьобщие ключевые узлы (понятия) в базе знаний. Интеграция многократно используемых компонентов ostis-систем сводится к отождествлению (склеиванию) ключевых узлов по различным признакам и устранению возможных дублирований и противоречий исходя из спецификации компонента и его содержания. Такой способ интеграции компонентов позволяет разрабатывать их параллельно и независимо друг от друга, что значительно сокращает сроки проектирования. Отождествление sc-элементов происходит в ходе выполнения \scnkeyword{действие. отождествить два указанных sc-элемента}. Автоматическая интеграция компонентов интеллектуальных систем представляет широкие возможности для существенного сокращения сроков проектирования интеллектуальных систем, поскольку позволяет использовать опыт прошлых разработок. Интеграция любых компонентов ostis-систем происходит автоматически, без вмешательства разработчика. Это достигается за счет использования SC-кода и его преимуществ, унификации спецификации многократно используемых компонентов и тщательного отбора компонентов в библиотеках экспертным сообществом, ответственным	за эту библиотеку.}
	 \begin{scnrelfromlist}{источник}
	 	\scnitem{\cite{Ivashenko2011}}
	 	\scnitem{\cite{Ivashenko2013}}
	 	\scnitem{\cite{Golenkov2014b}}
	 \end{scnrelfromlist}
	 \begin{scnindent}
	 	\scntext{примечание}{Это достигается за счёт использования \textit{SC-кода} и его преимуществ, унификации спецификации многократно используемых компонентов и тщательного отбора компонентов в библиотеках экспертным сообществом, ответственным за эту библиотеку.}
	 \end{scnindent}
	 \scnheader{ostis-система}
	 \scnsuperset{материнская ostis-система}
	 \begin{scnindent}
	 	\scntext{пояснение}{ostis-система, имеющая в своем составе библиотеку многократно используемых компонентов.}
	 	\scnhaselement{Метасистема OSTIS}
	 	\scntext{примечание}{материнская ostis-система в свою очередь может являться дочерней ostis-системой для какой-либо другой ostis-системы, заимствуя компоненты из библиотеки, входящей в состав этой другой ostis-системы.}
	 	\scntext{примечание}{материнская ostis-система отвечает за какой-то класс компонентов и является САПРом для этого класса, например, содержит методики разработки таких компонентов, их классификацию, подробные пояснения ко всем подклассам компонентов. Таким образом, формируется иерархия \scnkeyword{материнских ostis-систем}.}
	 \end{scnindent}
	 \scnsuperset{дочерняя ostis-система}
	 \begin{scnindent}
	 	\scntext{пояснение}{ostis-система, в составе которой имеется компонент, заимствованный из какой-либо библиотеки многократно используемых компонентов.}
	 \end{scnindent}
	 \scnheader{библиотека многократно используемых компонентов ostis-систем}
	 \begin{scnreltoset}{объединение}
	 	\scnitem{библиотека многократно используемых компонентов баз знаний ostis-систем}
	 	\scnitem{библиотека многократно используемых компонентов решателей задач ostis-систем}
	 	\scnitem{библиотека многократно используемых компонентов интерфейсов ostis-систем}
	 	\scnitem{библиотека встраиваемых ostis-систем}
	 	\scnitem{библиотека ostis-платформ}
	 \end{scnreltoset}
	 \scntext{примечание}{библиотека многократно используемых компонентов ostis-систем является подсистемой ostis-систем, которая имеет свою базу знаний, свой решатель задач и свой интерфейс. Однако не каждая ostis-система обязана иметь библиотеку компонентов.}
	 \begin{scnrelfromset}{обобщенная декомпозиция}
	 	\scnitem{база знаний библиотеки многократно используемых компонентов ostis-систем}
	 	\begin{scnindent}
	 		\scntext{примечание}{база знаний библиотеки многократно используемых компонентов ostis-систем представляет собой иерархию многократно используемых компонентов ostis-систем и их спецификаций.}
	 	\end{scnindent}
	 	\scnitem{решатель задач библиотеки многократно используемых компонентов ostis-систем}
	 	\begin{scnindent}
	 		\scntext{примечание}{решатель задач библиотеки многократно используемых компонентов ostis-систем реализует функциональные возможности библиотеки ostis-систем.}
	 	\end{scnindent}
	 	\scnitem{интерфейс библиотеки многократно используемых компонентов ostis-систем}
	 	\begin{scnindent}
	 		\scntext{примечание}{\scnkeyword{интерфейс библиотеки многократно используемых компонентов ostis-систем} обеспечивает доступ к многократно используемым компонентам и возможностям библиотеки ostis-систем для пользователя и других систем.}
	 		\begin{scnindent}
	 			\scnrelfrom{источник}{\cite{Koronchik2011}}
	 		\end{scnindent}
	 		\begin{scnrelfromset}{декомпозиция}
	 			\scnitem{минимальный интерфейс библиотеки многократно используемых компонентов ostis-систем}
	 			\begin{scnindent}
	 				\scntext{примечание}{Данный вид интерфейса позволяет \textit{менеджеру многократно используемых компонентов ostis-систем}, входящему в состав какой-либо дочерней ostis-системы, подключиться к библиотеке многократно используемых компонентов ostis-систем и использовать ее функциональные возможности, то есть, например, получить доступ к спецификации компонентов и установить выбранные компоненты в дочернюю ostis-систему, получить сведения о доступных версиях компонента, его зависимостях и так далее.}
	 			\end{scnindent}
	 			\scnitem{расширенный интерфейс библиотеки многократно используемых компонентов ostis-систем}
	 			\begin{scnindent}
	 				\scnidtf{графический интерфейс библиотеки многократно используемых компонентов ostis-систем}
	 				\scntext{примечание}{В частном случае у библиотеки может быть расширенный пользовательский интерфейс, который, в отличие от минимального интерфейса, позволяет не только получить доступ к компонентам для дальнейшей работы с ними, но и просматривать существующую структуру библиотеки, а также компоненты и их элементы в удобном и интуитивно понятном для пользователя виде.}
	 			\end{scnindent}
	 		\end{scnrelfromset}
	 	\end{scnindent}
	 \end{scnrelfromset}
	
		\bigskip
	\end{scnsubstruct}
	\scnsourcecomment{Завершили \scnqqi{Сегмент. Понятие библиотеки многократно используемых компонентов ostis-систем}}
\end{SCn}