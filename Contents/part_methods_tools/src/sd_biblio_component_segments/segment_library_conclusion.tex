\begin{SCn}
\scnsectionheader{Сегмент. Заключение в Предметную область и онтологию комплексной библиотеки многократно используемых семантически совместимых компонентов ostis-систем}
	
\begin{scnsubstruct}
	\scnheader{компонентное проектирование интеллектуальных компьютерных систем}  
	\scntext{примечание}{Компонентный подход является ключевым в технологии проектирования интеллектуальных компьютерных систем. Вместе с этим, технология компонентного проектирования тесно связана с остальными составляющими \textit{технологии проектирования интеллектуальных компьютерных систем} и обеспечивает их совместимость, производя мощнейший синергетический эффект при использовании всего комплекса частных технологий проектирования интеллектуальных систем. Важнейшим принципом в реализации компонентного подхода является семантическая совместимость многократно используемых компонентов, что позволяет минимизировать участие программистов в создании новых компьютерных систем и в совершенствовании существующих компьютерных систем.}
	\scntext{примечание}{Для реализации компонентного подхода предлагается \textit{библиотека многократно используемых совместимых компонентов интеллектуальных компьютерных систем на основе Технологии OSTIS}, введена классификация и спецификация многократно используемых компонентов ostis-систем, предложена модель менеджера компонентов, позволяющего взаимодействовать \textit{ostis-системам} с \textit{библиотеками многократно используемых компонентов} и управлять компонентами в системе, рассмотрена архитектура экосистемы \textit{интеллектуальных компьютерных систем} с точки зрения использования библиотеки многократно используемых компонентов.}
	\scntext{примечание}{Полученные результаты позволят повысить эффективность проектирования интеллектуальных систем и средств автоматизации разработки таких систем, а также обеспечить возможность не только разработчику, но и интеллектуальной системе автоматически дополнять систему новыми знаниями и навыками.}
		\bigskip
\end{scnsubstruct}
\scnsourcecomment{Завершили \scnqqi{Сегмент. Заключение в Предметную область и онтологию комплексной библиотеки многократно используемых семантически совместимых компонентов ostis-систем}}
\end{SCn}