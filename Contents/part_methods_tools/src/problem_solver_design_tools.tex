\begin{SCn}
\scnsectionheader{Логико-семантическая модель комплекса ostis-систем автоматизации проектирования решателей задач ostis-систем}
\begin{scnrelfromvector}{введение}
	\scnfileitem{К числу задач системы автоматизации проектирования \textit{решателей задач} \textit{ostis-систем} относится техническая поддержка разработчиков решателей, в том числе --- обеспечение корректного и эффективного выполнения этапов, предусмотренных \textit{Методикой проектирования машины обработки знаний ostis-систем}.}
		
	\scnfileitem{При разработке любых компонентов ostis-систем используются схожие принципы. Одним из основных принципов является принцип использования готовых компонентов различного рода, уже имеющихся в библиотеке многократно используемых компонентов ostis-систем, входящей в состав \textit{Метасистемы OSTIS}.}
\end{scnrelfromvector}

\begin{scnreltovector}{конкатенация сегментов}
	\scnitem{Сегмент. Семантическая модель базы знаний системы автоматизации проектирования решателей задач ostis-систем}
	\scnitem{Сегмент. Семантическая модель решателя задач системы автоматизации проектирования решателей задач ostis-систем}
	\scnitem{Сегмент. Семантическая модель пользовательского интерфейса системы автоматизации проектирования решателей задач ostis-систем}
\end{scnreltovector}

\begin{scnrelfromlist}{ключевой знак}
	\scnitem{Система автоматизации проектирования решателей задач ostis-систем}
	\scnitem{Решатель задач системы автоматизации проектирования агентов обработки знаний}
	\scnitem{Решатель задач системы автоматизации проектирования scp-программ}
\end{scnrelfromlist}

\begin{scnrelfromlist}{ключевое понятие}
	\scnitem{решатель задач ostis-системы}
	\scnitem{sc-агент}
	\scnitem{база знаний}
	\scnitem{scp-программа}
	\scnitem{точка останова*}
	\scnitem{точка останова}
	\scnitem{некорректность в scp-программе*}
	\scnitem{некорректность в scp-программе}
	\scnitem{ошибка в scp-программе}
	\scnitem{библиотека многократно используемых компонентов ostis-систем}
	\scnitem{команда пользовательского интерфейса системы автоматизации проектирования решателей задач ostis-систем}
	\scnitem{команда пользовательского интерфейса системы автоматизации проектирования агентов обработки знаний}
	\scnitem{команда пользовательского интерфейса системы автоматизации проектирования scp-программ}
\end{scnrelfromlist}

\begin{scnsubstruct}
	\scnheader{Система автоматизации проектирования решателей задач ostis-систем} 
	\scnnote{\textit{Система автоматизации проектирования решателей задач ostis-систем} сама по себе также является ostis-системой и имеет соответствующую структуру. Таким образом, модель данной системы включает \textit{sc-модель базы знаний}, \textit{sc-модель объединенного решателя задач} и \textit{sc-модель пользовательского интерфейса}.}
	\begin{scnrelfromlist}{подсистема}
		\scnitem{подсистема автоматизации проектирования агентов обработки знаний (sc-агентов)}
		\scnitem{подсистема автоматизации проектирования scp-программ}
	\end{scnrelfromlist}
	
	\begin{scnrelfromset}{способы использования}
		\scnfileitem{Подсистема в рамках \textit{Метасистемы OSTIS}.}
			\begin{scnindent}
				\scntext{пояснение}{Данный вариант использования предполагает отладку необходимых компонентов в рамках \textit{Метасистемы OSTIS} с последующим переносом их в \textit{дочернюю ostis-систему}.}
				\begin{scnindent}
					\scnrelfrom{смотрите}{Предметная область и онтология комплексной библиотеки многократно используемых семантически совместимых компонентов ostis-систем}						
				\end{scnindent}
			\end{scnindent}
		
		\scnfileitem{Cамостоятельная ostis-система, предназначенная исключительно для разработки и отладки компонентов решателей задач.}
		\begin{scnindent}
			\scnnote{В этом случае проектируемые компоненты отлаживаются в рамках такой системы, а затем должны быть перенесены в дочернюю ostis-систему.}						
		\end{scnindent}
		
		\scnfileitem{Подсистема в рамках дочерней ostis-системы.}
		\begin{scnindent}
			\scnnote{В таком варианте отладка компонентов осуществляется непосредственно в той же системе, в которой предполагается их использование, и дополнительного переноса не требуется.}						
		\end{scnindent}
	\end{scnrelfromset}
	\scnnote{Независимо от выбранного способа использования системы, разработанные компоненты впоследствии могут быть включены в состав \textit{библиотеки многократно используемых компонентов ostis-систем}.}
	
	\scnheader{уровень отладки \textit{решателя задач}}
	\begin{scneqtoset}
		\scnitem{отладка на уровне sc-агентов}
		\begin{scnindent}
			\scnnote{В случае отладки на уровне sc-агентов акт выполнения каждого агента считается неделимым и не может быть прерван. При этом может выполняться отладка как атомарных sc-агентов, так и неатомарных. Инициирование того или иного агента, в том числе входящего в состав неатомарного, осуществляется путем создания соответствующих конструкций в sc-памяти, таким образом, отладка может осуществляться на разных уровнях детализации агентов, вплоть до атомарных.}
		\end{scnindent}
		\scnitem{отладка на уровне scp-программ}
		\begin{scnindent}
			\scnnote{Отладка на уровне \textit{scp-программ} осуществляется аналогично существующим современным подходам к отладке процедурных программ и предполагает возможность установки точек останова, пошагового выполнения программы и так далее.}
		\end{scnindent}
	\end{scneqtoset}

	\scnnote{Система поддержки проектирования агентов может служить основой для моделирования систем агентов, использующих другие принципы коммуникации, например, непосредственный обмен сообщениями между агентами.}
	
	\scnheader{Система автоматизации проектирования решателей задач ostis-систем}
	\begin{scnrelfromset}{базовая декомпозиция}
		\scnitem{Система автоматизации проектирования агентов обработки знаний}
		\begin{scnindent}
			\begin{scnrelfromset}{базовая декомпозиция}
				\scnitem{База знаний системы автоматизации проектирования агентов обработки знаний}
				\scnitem{Решатель задач системы автоматизации проектирования агентов обработки знаний}
				\scnitem{Пользовательский интерфейс системы автоматизации проектирования агентов обработки знаний}
			\end{scnrelfromset}
		\end{scnindent}
		\scnitem{Система автоматизации проектирования scp-программ}
		\begin{scnindent}
			\begin{scnrelfromset}{базовая декомпозиция}
				\scnitem{База знаний системы автоматизации проектирования scp-программ}
				\scnitem{Решатель задач системы автоматизации проектирования scp-программ}
				\scnitem{Пользовательский интерфейс системы автоматизации проектирования scp-программ}
			\end{scnrelfromset}
		\end{scnindent}
	\end{scnrelfromset}

	\scnsectionheader{Сегмент. Семантическая модель базы знаний системы автоматизации проектирования решателей задач ostis-систем}
	\begin{scnsubstruct}
	\begin{scnrelfromlist}{ключевое понятие}
		\scnitem{решатель задач ostis-системы}
		\scnitem{scp-программа}
		\scnitem{точка останова*}
		\scnitem{точка останова}
		\scnitem{некорректность в scp-программе*}
		\scnitem{некорректность в scp-программе}
		\scnitem{ошибка в scp-программе}
	\end{scnrelfromlist}
	\scnnote{\textit{База знаний системы автоматизации проектирования решателей задач ostis-систем} включает в себя кроме Ядра баз знаний ostis-систем и Предметной области Базового языка программирования ostis-систем также описание ключевых понятий, связанных с верификацией и отладкой \textit{scp-программ}.}
	
	\scnheader{точка останова*}
	\scniselement{квазибинарное отношение}
	\scnrelfrom{первый домен}{scp-программа}
	\scnrelfrom{второй домен}{множество sc-переменных}
	\begin{scnindent}
		\scnnote{Данные sc-переменные должны соответствовать \textit{scp-операторам} в рамках этой программы.}			
	\end{scnindent}
	\scnnote{При генерации каждого \textit{scp-процесса}, соответствующего этой \textit{scp-программе}, все \textit{scp-операторы}, соответствующие таким переменным, будут добавлены во множество \textit{точка останова}, то есть выполнение данного scp-процесса будет прерываться при достижении каждого из этих \textit{scp-операторов}. Использование данного отношения приводит к указанию точек останова для всех \textit{scp-процессов}, формируемых на основе заданной \textit{\mbox{scp-программы}}. Для указания точки останова в рамках отдельно взятого \textit{scp-процесса} нужный scp-оператор явно включается во множество \textit{точка останова}.}
	\scnnote{Во множество \textbf{\textit{точка останова}} входят все \textit{scp-операторы}, являющиеся точками останова в рамках какого-либо \textit{scp-процесса}. Это означает, что в момент, когда в соответствии с переходами между \textit{scp-операторами} по связкам отношения \textit{последовательность действий*} указанный \textit{scp-оператор} должен стать \textit{активным действием}, он становится \textit{отложенным действием}, и, соответственно, выполнение всего \textit{scp-процесса} по данной ветке приостанавливается. Чтобы продолжить выполнение, необходимо удалить указанный \textit{\mbox{scp-оператор}} из множества \textit{отложенных действий} и добавить его во множество \textit{активных действий}.}
	
	\scnheader{точка останова}
	\scnsubset{scp-оператор}
	
	\scnheader{некорректность в scp-программе}
	\scnsubset{некорректная структура}
	\scnsuperset{ошибка в scp-программе}
	\scnsuperset{недостижимый scp-оператор}
	\scnsuperset{потенциально бесконечный цикл}
	\scntext{пояснение}{Под \textit{некорректностью в scp-программе} понимается \textit{некорректная структура}, описывающая некорректность (не обязательно делающую невозможным выполнение соответствующих данной \textit{scp-программе scp-процессов}), выявленную в рамках какой-либо конкретной \textit{scp-программы}.}
	
	\scnheader{ошибка в scp-программе}
	\scntext{пояснение}{Под \textit{ошибкой в scp-программе} понимается такая \textit{некорректность в scp-программе}, которая делает невозможным успешное выполнение любого \textit{scp-процесса}, соответствующего данной \textit{scp-программе}, или даже создание такого \textit{scp-процесса}.}
	\begin{scnrelfromset}{разбиение}
		\scnitem{синтаксическая ошибка в scp-программе}
		\begin{scnindent}
			\scnexplanation{Под \textit{синтаксической ошибкой в scp-программе} понимается \textit{ошибка в scp-программе}, в состав которой входит некоторая конструкция, не соответствующая синтаксису \textit{scp-программы} или какой-либо ее части, например, конкретного \textit{scp-оператора}.}
		\end{scnindent}
		\scnitem{семантическая ошибка в scp-программе}
		\begin{scnindent}
			\scnexplanation{Под \textit{семантической ошибкой в scp-программе} понимается \textit{ошибка в scp-программе}, в состав которой входит некоторая конструкция, корректная с точки зрения синтаксиса, но некорректная с семантической точки зрения, например, нарушающая логическую целостность \textit{scp-программы}.}
		\end{scnindent}
	\end{scnrelfromset}
	\begin{scnrelfromset}{разбиение}
		\scnitem{ошибка в scp-программе на уровне программы}
		\scnitem{ошибка в scp-программе на уровне множества параметров}
		\scnitem{ошибка в scp-программе на уровне множества операторов}
		\scnitem{ошибка в scp-программе на уровне оператора}
		\scnitem{ошибка в scp-программе на уровне операнда}
	\end{scnrelfromset}
	\scnnote{Каждая \textit{ошибка в scp-программе на уровне программы} описывает некорректный фрагмент, выявление которого требует анализа всей \textit{scp-программы} как единого целого, и не может быть выполнено путем анализа ее отдельных частей, например, конкретных \textit{scp-операторов}.}
	
	\scnheader{ошибка в scp-программе на уровне программы}
	\scnsuperset{отсутствует scp-процесс, соответствующий данной scp-программе}
	\begin{scnindent}
		\scniselement{синтаксическая ошибка в scp-программе}
	\end{scnindent}
	\scnsuperset{не указана декомпозиция scp-процесса, соответствующего данной scp-программе}
	\begin{scnindent}
		\scniselement{синтаксическая ошибка в scp-программе}
	\end{scnindent}
	\scnnote{Каждая \textit{ошибка в scp-программе на уровне множества параметров} описывает некорректный фрагмент, для выявления которого достаточно анализа параметров некоторой \textit{scp-программы}, то есть явным образом выделенных аргументов \textit{действия (scp-процессе)}, соответствующего данной scp-программе. К такого рода ошибкам относятся, например, неверное указание ролей этих аргументов в рамках данного действия.}
	
	\scnheader{ошибка в scp-программе на уровне множества параметров}
	\scnsuperset{не указан тип параметра scp-программы}
	\begin{scnindent}
		\scniselement{синтаксическая ошибка в scp-программе}
	\end{scnindent}
	\scnsuperset{не указан порядковый номер параметра scp-программы}
	\begin{scnindent}
		\scniselement{синтаксическая ошибка в scp-программе}
	\end{scnindent}
	\scnnote{Каждая \textit{ошибка в scp-программе на уровне множества операторов} описывает некорректный фрагмент, для выявления которого достаточно анализа множества операторов некоторой \textit{scp-программы}, то есть элементов декомпозиции \textit{действия (scp-процесса)}, соответствующего данной \textit{scp-программе}. К таким ошибкам относится, например, факт отсутствия \textit{начального оператора' scp-программы} или факт отсутствия в программе \textit{scp-оператора завершения выполнения программы}.}
	
	\scnheader{ошибка в scp-программе на уровне множества операторов}
	\scnsuperset{декомпозиция scp-процесса не содержит ни одного элемента}
	\begin{scnindent}
		\scniselement{синтаксическая ошибка в scp-программе}
	\end{scnindent}
	\scnsuperset{отсутствует scp-оператор завершения выполнения программы}
	\begin{scnindent}
		\scniselement{синтаксическая ошибка в scp-программе}
	\end{scnindent}
	\scnsuperset{scp-оператор, к которому осуществляется переход, не является частью текущего scp-процесса}
	\begin{scnindent}
		\scniselement{синтаксическая ошибка в scp-программе}
	\end{scnindent}
	\scnsuperset{не указана последовательность действий после выполнения текущего scp-оператора}
	\begin{scnindent}
		\scniselement{синтаксическая ошибка в scp-программе}
	\end{scnindent}
	\scnsuperset{отсутствует начальный оператор scp-программы}
	\begin{scnindent}
		\scniselement{синтаксическая ошибка в scp-программе}
	\end{scnindent}
	\scnnote{Каждая \textit{ошибка в scp-программе на уровне оператора} описывает некорректный фрагмент, для выявления которого достаточно анализа одного конкретного \textit{scp-оператора}, при этом не важно, в состав какой \textit{scp-программы} он входит. К такого рода ошибкам относится, например, факт указания количества операндов \textit{scp-оператора}, не соответствующего спецификации соответствующего класса \textit{scp-операторов}.}
	
	\scnheader{ошибка в scp-программе на уровне оператора}
	\scnsuperset{scp-оператор не принадлежит ни одному из атомарных классов scp-операторов}
	\begin{scnindent}
		\scniselement{синтаксическая ошибка в scp-программе}
	\end{scnindent}
	\scnsuperset{ни один операнд scp-оператора удаления не помечен как удаляемый sc-элемент}
	\begin{scnindent}
		\scniselement{синтаксическая ошибка в scp-программе}
	\end{scnindent}
	\scnsuperset{в scp-операторе поиска пятиэлементной конструкции совпадает второй и четвертый операнд}
	\begin{scnindent}
		\scniselement{синтаксическая ошибка в scp-программе}
	\end{scnindent}
	\scnsuperset{scp-оператор поиска не содержит ни одного операнда с заданным значением}
	\begin{scnindent}
		\scniselement{синтаксическая ошибка в scp-программе}
	\end{scnindent}
	\scnsuperset{scp-оператор поиска с формированием множеств не содержит ни одного операнда с атрибутом формируемое множество}
	\begin{scnindent}
		\scniselement{синтаксическая ошибка в scp-программе}
	\end{scnindent}
	\scnsuperset{атрибутом формируемое множество отмечен операнд, которому соответствует операнд с заданным значением}
	\begin{scnindent}
		\scniselement{синтаксическая ошибка в scp-программе}
	\end{scnindent}
	\scnsuperset{количество операндов scp-оператора не совпадает со спецификацией}
	\begin{scnindent}
		\scniselement{синтаксическая ошибка в scp-программе}
	\end{scnindent}
	\scnnote{Каждая \textit{ошибка в scp-программе на уровне операнда} описывает некорректный фрагмент, для выявления которого достаточно анализа одного конкретного операнда в рамках scp-программы, точнее sc-дуги принадлежности, связывающей указанный операнд и соответствующий \textit{scp-оператор}, при этом не важно, какой именно \textit{scp-оператор}. К такого рода ошибкам относится, например, факт отсутствия ролевого отношения, указывающего на номер операнда в рамках \textit{scp-оператора}.}
	
	\scnheader{ошибка в scp-программе на уровне операнда}
	\scnsuperset{не указан номер операнда в рамках scp-оператора}
	\begin{scnindent}
		\scniselement{синтаксическая ошибка в scp-программе}
	\end{scnindent}
		
	\scnheader{некорректность в scp-программе*}
	\scniselement{бинарное отношение}
	\scnrelfrom{первый домен}{некорректность в scp-программе}
	\scnrelfrom{второй домен*}{scp-программа}
	\scnnote{Отношение \textit{scp-программа поиска некорректности в scp-программе*} связывает \textit{класс некорректностей в scp-программе} и \textit{scp-программу}, которая может использоваться для выявления соответствующей некорректности в какой-либо другой \textit{scp-программе}. Указанная \textit{scp-программа} должна иметь единственный параметр, который является \textit{in-параметром’} и, в зависимости от соответствующего класса некорректностей в \textit{scp-программе}, обозначает:
	\begin{scnitemize}
		\item саму \textit{scp-программу} в случае выявления \textit{некорректности в scp-программе} вообще или \textit{ошибки в scp-программе на уровне программы};
		\item \textit{scp-процесс}, являющийся \textit{ключевым sc-элементом} данной \textit{scp-программы} в случае выявления ошибки в \textit{scp-программе на уровне множества параметров};
		\item \textit{множество операторов} данной \textit{scp-программы} в случае выявления \textit{ошибки в scp-программе на уровне множества операторов};
		\item \textit{знак конкретного scp-оператора} в случае выявления ошибки в \textit{scp-программе на уровне оператора};
		\item \textit{sc-дугу принадлежности} в случае выявления \textit{ошибки в scp-программе на уровне операнда}.
	\end{scnitemize}
	}
	\scnnote{Если в результате верификации \textit{scp-программы} выявлена некорректность, то формируется соответствующая \textit{структура} и генерируется связка отношения \textit{некорректность в scp-программе*}.}
	\end{scnsubstruct}
	

	\scnsectionheader{Сегмент. Семантическая модель решателя задач системы автоматизации проектирования решателей задач ostis-систем}
	\begin{scnsubstruct}
		\begin{scnrelfromlist}{ключевое понятие}
			\scnitem{Решатель задач системы автоматизации проектирования агентов обработки знаний}
			\scnitem{Решатель задач системы автоматизации проектирования scp-программ}
		\end{scnrelfromlist}
		
	\scnheader{Решатель задач системы автоматизации проектирования агентов обработки знаний}
	\begin{scnrelfromset}{декомпозиция}
		\scnitem{Множеcтво методов решателя задач системы автоматизации проектирования агентов обработки знаний}
		\scnitem{Машина обработки знаний системы автоматизации проектирования агентов обработки знаний}
		\begin{scnindent}
			\begin{scnrelfromset}{декомпозиция абстрактного sc-агента}
				\scnitem{Абстрактный sc-агент верификации sc-агентов}
				\begin{scnindent}
					\begin{scnrelfromset}{декомпозиция абстрактного sc-агента}
						\scnitem{Абстрактный sc-агент верификации спецификации sc-агента}
						\scnitem{Абстрактный sc-агент проверки неатомарного sc-агента на непротиворечивость его спецификации спецификациям более частных sc-агентов в его составе}
					\end{scnrelfromset}
				\end{scnindent}
				\scnitem{Абстрактный sc-агент отладки коллективов sc-агентов}
				\begin{scnindent}
					\begin{scnrelfromset}{декомпозиция абстрактного sc-агента}
						\scnitem{Абстрактный sc-агент поиска всех выполняющихся процессов, соответствующих заданному sc-агенту}
						\begin{scnindent}
							\scnnote{Единственным аргументом класса действий, соответствующего \textit{Абстрактному sc-агенту поиска всех выполняющихся процессов, соответствующих заданному sc-агенту} является знак этого \textit{sc-агента}.}
						\end{scnindent}
						\scnitem{Абстрактный sc-агент инициирования заданного sc-агента на заданных аргументах}
						\begin{scnindent}
							\scnnote{Класс действий, соответствующий \textit{Абстрактному sc-агенту инициирования заданного sc-агента на заданных аргументах}, имеет два аргумента. Первый аргумент является знаком инициируемого sc-агента, второй --- знаком связки, в которую под соответствующими атрибутами входят sc-элементы, которые станут аргументами соответствующего \textit{действия в sc-памяти}.}
						\end{scnindent}
						\scnitem{Абстрактный sc-агент активации заданного sc-агента}
						\begin{scnindent}
							\scnnote{Единственным аргументом класса действий, соответствующего \textit{Абстрактному sc-агенту активации заданного sc-агента} является знак этого \textit{sc-агента}.}
						\end{scnindent}
						\scnitem{Абстрактный sc-агент деактивации заданного sc-агента}
						\begin{scnindent}
							\scnnote{Единственным аргументом класса действий, соответствующего \textit{Абстрактному sc-агенту деактивации заданного sc-агента}, является знак этого \textit{sc-агента}.}
						\end{scnindent}
						\scnitem{Абстрактный sc-агент установки блокировки заданного типа для заданного процесса на заданный sc-элемент}
						\begin{scnindent}														
							\scnnote{Класс действий, соответствующий \textit{Абстрактному sc-агенту установки блокировки заданного типа на заданный sc-элемент}, имеет три аргумента. Первый аргумент является знаком класса блокировок, второй --- знаком процесса в sc-памяти, третий --- sc-элементом, на который должна быть установлена блокировка.}
						\end{scnindent}
						
						\scnitem{Абстрактный sc-агент снятия всех блокировок заданного процесса}
						\begin{scnindent}														
							\scnnote{Единственным аргументом класса действий, соответствующего \textit{Абстрактному sc-агенту снятия всех блокировок заданного процесса}, является знак этого \textit{процесса в sc-памяти}.}
						\end{scnindent}
						\scnitem{Абстрактный sc-агент снятия всех блокировок с заданного sc-элемента}
						\begin{scnindent}														
							\scnnote{Единственным аргументом класса действий, соответствующего \textit{Абстрактному sc-агенту снятия всех блокировок с заданного sc-элемента}, является знак этого \textit{sc-элемента}.}
						\end{scnindent}
					\end{scnrelfromset}
				\end{scnindent}
			\end{scnrelfromset}
		\end{scnindent}
	\end{scnrelfromset}
			
	\scnheader{Решатель задач системы автоматизации проектирования scp-программ}
	\begin{scnrelfromset}{декомпозиция}
		\scnitem{Множество методов решателя задач системы автоматизации проектирования агентов обработки знаний}
		\scnitem{Машина обработки знаний системы автоматизации проектирования агентов обработки знаний}
		\begin{scnindent}
			\begin{scnrelfromset}{декомпозиция абстрактного sc-агента}
				\scnitem{Абстрактный sc-агент верификации scp-программ}
				\begin{scnindent}														
					\scntext{обобщённый алгоритм}{Алгоритм работы \textit{абстрактного sc-агента верификации scp-программ} сводится к поиску некорректностей в рамках \textit{scp-программы} на основе определений, соответствующих различным классам таких некорректностей, а также посредством запуска соответствующих данным классам некорректностей \textit{scp-программ поиска некорректности в scp-программе*}.}
					\scntext{результат}{Результатом работы \textit{абстрактного sc-агента верификации scp-программ} является формирование в \textit{sc-памяти структур}, описывающих некорректности в исследуемой \textit{scp-программе}, если таковые имеются.}
					\scnnote{Единственным аргументом класса действий, соответствующего \textit{абстрактному sc-агенту верификации scp-программ}, является знак верифицируемой \textit{scp-программы}.}
				\end{scnindent}
				\scnitem{Абстрактный sc-агент отладки scp-программ}
				\begin{scnindent}
					\begin{scnrelfromset}{декомпозиция абстрактного sc-агента}
						\scnitem{Абстрактный sc-агент запуска заданной scp-программы для заданного множества входных данных}
						\begin{scnindent}														
						\scnnote{Класс действий, соответствующий \textit{абстрактному sc-агенту запуска заданной scp-программы для заданного множества входных данных} имеет два аргумента. Первый аргумент является знаком запускаемой scp-программы, второй --- знаком связки, в которую под соответствующими атрибутами входят sc-элементы, которые станут аргументами соответствующего scp-процесса.}
						\scnnote{В режиме пошагового выполнения предполагается, что на каждом шаге инициируется выполнение всех scp-операторов в рамках заданного scp-процесса, для которых предыдущий scp-оператор стал прошлой сущностью (выполнился). В свою очередь, шаг заканчивается, когда все инициированные таким образом операторы закончат выполнение. Таким образом, в случае, если в рамках scp-программы есть параллельные ветви, то на одном шаге могут одновременно инициироваться два и более scp-оператора.}
						\end{scnindent}
						\scnitem{Абстрактный sc-агент запуска заданной scp-программы для заданного множества входных данных в режиме пошагового выполнения}
						\begin{scnindent}														
							\scnnote{Класс действий, соответствующий \textit{абстрактному sc-агенту запуска заданной scp-программы для заданного множества входных данных в режиме пошагового выполнения} имеет два аргумента. Первый аргумент является знаком запускаемой scp-программы, второй --- знаком связки, в которую под соответствующими атрибутами входят sc-элементы, которые станут аргументами соответствующего scp-процесса.}
							\scnnote{В режиме пошагового выполнения предполагается, что на каждом шаге инициируется выполнение всех scp-операторов в рамках заданного scp-процесса, для которых предыдущий scp-оператор стал прошлой сущностью (выполнился). В свою очередь, шаг заканчивается, когда все инициированные таким образом операторы закончат выполнение. Таким образом, в случае, если в рамках scp-программы есть параллельные ветви, то на одном шаге могут одновременно инициироваться два и более scp-оператора.}
						\end{scnindent}
						\scnitem{Абстрактный sc-агент поиска всех scp-операторов в рамках scp-программы}
						\begin{scnindent}
							\scnnote{Единственным аргументом класса действий, соответствующего \textit{абстрактному sc-агенту поиска всех scp-операторов в рамках scp-программы}, является знак этой scp-программы.}
						\end{scnindent}
						\scnitem{Абстрактный sc-агент поиска всех точек останова в рамках scp-процесса}
						\begin{scnindent}														
							\scnnote{Единственным аргументом класса действия, соответствующего \textit{абстрактному sc-агенту поиска всех точек останова в рамках scp-процесса}, является знак scp-процесса, с которым будет выполнено соответствующее действие.}
						\end{scnindent}
						\scnitem{Абстрактный sc-агент добавления точки останова в scp-программу}
						\begin{scnindent}														
							\scnnote{Класс действий, соответствующий \textit{абстрактному sc-агенту добавления точки останова в scp-программу} имеет два аргумента. Первый аргумент является знаком scp-программы или scp-процесса соответственно, второй --- знаком scp-оператора, входящего в состав этой scp-программы или scp-процесса.}
						\end{scnindent}
						\scnitem{Абстрактный sc-агент удаления точки останова из scp-программы}
						\begin{scnindent}														
							\scnnote{Класс действий, соответствующий \textit{абстрактному sc-агенту удаления точки останова из scp-программы} имеет два аргумента. Первый аргумент является знаком scp-программы или scp-процесса соответственно, второй --- знаком scp-оператора, входящего в состав этой scp-программы или scp-процесса.}
						\end{scnindent}
						\scnitem{Абстрактный sc-агент добавления точки останова в scp-процесс}
						\begin{scnindent}														
							\scnnote{Класс действий, соответствующий \textit{абстрактному sc-агенту добавления точки останова в scp-процесс} имеет два аргумента. Первый аргумент является знаком scp-программы или scp-процесса соответственно, второй --- знаком scp-оператора, входящего в состав этой scp-программы или scp-процесса.}
						\end{scnindent}
						\scnitem{Абстрактный sc-агент удаления точки останова из scp-процесса}
						\begin{scnindent}														
							\scnnote{Класс действий, соответствующий \textit{абстрактному sc-агенту удаления точки останова из scp-процесса} имеет два аргумента. Первый аргумент является знаком scp-программы или scp-процесса соответственно, второй --- знаком scp-оператора, входящего в состав этой scp-программы или scp-процесса.}
						\end{scnindent}
						\scnitem{Абстрактный sc-агент продолжения выполнения scp-процесса на один шаг}
						\begin{scnindent}														
							\scnnote{Единственным аргументом класса действия, соответствующего \textit{абстрактному sc-агенту продолжения выполнения scp-процесса на один шаг}, является знак scp-процесса, с которым будет выполнено соответствующее действие.}
						\end{scnindent}
						\scnitem{Абстрактный sc-агент продолжения выполнения scp-процесса до точки останова или завершения}
						\begin{scnindent}														
							\scnnote{Единственным аргументом класса действия, соответствующего \textit{абстрактному sc-агенту продолжения выполнения scp-процесса до точки останова или завершения}, является знак scp-процесса, с которым будет выполнено соответствующее действие.}
						\end{scnindent}
						\scnitem{Абстрактный sc-агент просмотра информации об scp-процессе}
						\begin{scnindent}														
							\scnnote{Единственным аргументом класса действия, соответствующего \textit{абстрактному sc-агенту просмотра информации об scp-процессе}, является знак scp-процесса, с которым будет выполнено соответствующее действие.}
						\end{scnindent}
						\scnitem{Абстрактный sc-агент просмотра информации об scp-операторе}
						\begin{scnindent}														
							\scnnote{Единственным аргументом класса действий, соответствующего \textit{абстрактному sc-агенту просмотра информации об scp-операторе}, является знак scp-оператора, входящего в состав некоторого scp-процесса.}
							\scntext{результат}{Результатом работы данного агента является структура, описывающая значения операндов данного scp-оператора, его атомарный тип и другую служебную информацию, полезную для разработчика.}
						\end{scnindent}
					\end{scnrelfromset}
				\end{scnindent}
			\end{scnrelfromset}
		\end{scnindent}
	\end{scnrelfromset}
	\end{scnsubstruct}
		

	\scnsectionheader{Сегмент. Семантическая модель пользовательского интерфейса системы автоматизации проектирования решателей задач ostis-систем}
	\begin{scnsubstruct}
	\begin{scnrelfromlist}{ключевое понятие}
		\scnitem{решатель задач ostis-системы}
		\scnitem{команда пользовательского интерфейса системы автоматизации проектирования решателей задач ostis-систем}
		\scnitem{команда пользовательского интерфейса системы автоматизации проектирования агентов обработки знаний}
		\scnitem{команда пользовательского интерфейса системы автоматизации проектирования scp-программ}
	\end{scnrelfromlist}
	\scnnote{\textit{пользовательский интерфейс системы автоматизации проектирования решателей задач ostis-систем} представлен набором интерфейсных команд, позволяющих пользователю инициировать деятельность нужного агента, входящего в состав этой системы.}
		
	\scnheader{команда пользовательского интерфейса системы автоматизации проектирования решателей задач ostis-систем}
	\begin{scnrelfromset}{разбиение}
		\scnitem{команда пользовательского интерфейса системы автоматизации проектирования агентов обработки знаний}
		\scnitem{команда пользовательского интерфейса системы автоматизации проектирования программ языка SCP}
	\end{scnrelfromset}
	
	\scnheader{команда пользовательского интерфейса системы автоматизации проектирования агентов обработки знаний}
	\begin{scnrelfromset}{разбиение}
		\scnitem{команда верификации sc-агентов}
		\begin{scnindent}
			\begin{scnrelfromset}{разбиение}
				\scnitem{команда верификации спецификации sc-агента}
				\scnitem{команда верификации неатомарного sc-агента на непротиворечивость его спецификации спецификациям более частных sc-агентов в его составе}
			\end{scnrelfromset}
		\end{scnindent}
		\scnitem{команда отладки коллективов sc-агентов}
		\begin{scnindent}
			\begin{scnrelfromset}{разбиение}
				\scnitem{команда поиска всех выполняющихся процессов, соответствующих заданному sc-агенту}
				\scnitem{команда инициирования заданного sc-агента на заданных аргументах}
				\scnitem{команда активации заданного sc-агента}
				\scnitem{команда деактивации заданного sc-агента}
				\scnitem{команда установки блокировки заданного типа для заданного процесса на заданный sc-элемент}
				\scnitem{команда снятия всех блокировок заданного процесса}
				\scnitem{команда снятия всех блокировок с заданного sc-элемента}
			\end{scnrelfromset}
		\end{scnindent}
	\end{scnrelfromset}
	
	\scnheader{команда пользовательского интерфейса системы автоматизации проектирования scp-программ}
		\begin{scnrelfromset}{разбиение}
			\scnitem{команда верификации scp-программ}
			\scnitem{команда отладки scp-программ}
			\begin{scnindent}
				\begin{scnrelfromset}{разбиение}
					\scnitem{команда запуска заданной scp-программы для заданного множества входных данных}
					\scnitem{команда запуска заданной scp-программы для заданного множества входных данных в режиме пошагового выполнения}
					\scnitem{команда поиска всех scp-операторов в рамках scp-программы}
					\scnitem{команда поиска всех точек останова в рамках scp-процесса}
					\scnitem{команда добавления точки останова в scp-программу}
					\scnitem{команда удаления точки останова из scp-программы}
					\scnitem{команда добавления точки останова в scp-процесс}
					\scnitem{команда удаления точки останова из scp-процесса}
					\scnitem{команда продолжения выполнения scp-процесса на один шаг}
					\scnitem{команда продолжения выполнения scp-процесса до точки останова или завершения}
					\scnitem{команда просмотра информации об scp-процессе}
					\scnitem{команда просмотра информации об scp-операторе}
				\end{scnrelfromset}
			\end{scnindent}
		\end{scnrelfromset}
	\end{scnsubstruct}

\end{scnsubstruct}

\end{SCn}
