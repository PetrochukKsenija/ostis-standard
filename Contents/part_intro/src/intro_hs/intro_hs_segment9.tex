\scnsegmentheader{Комплекс свойств, определяющих уровень интероперабельности
    кибернетической системы как фактора существенного повышения уровня ее
    обучаемости, а также фактора существенного повышения качества всех тех
    многоагентных систем, в состав которых входит данная кибернетическая система}

\begin{scnsubstruct}
    \scnidtf{Комплекс свойств \textit{кибернетической системы}, определяющих
        необходимые требования к тем \textit{кибернетическим системам}, которые могут
        входить в состав \textit{синергетических кибернетических систем}}
    \scnheader{интероперабельность кибернетической системы}
    \scnidtf{способность кибернетической системы взаимодействовать с другими
        кибернетическими системами в целях создания коллектива кибернетических систем
        (\textit{многоагентных систем}), уровень качества и, в частности, уровень
        \textit{интеллекта} которого выше уровня качества каждой
        \textit{кибернетической системы}, входящей в состав этого коллектива)}
    \scnidtf{комплекс способностей кибернетической системы, которые определяют ее
        вклад в уровень коллективной (социальной) интеллектуальности, т.е. в уровень
        интеллектуальности того коллектива кибернетических систем, членом которого
        данная кибернетическая система является (в уровень интеллектуальности
        соответствующей многоагентной системы)}
    \scnidtf{уровень вклада \textit{кибернетической системы} в обеспечение
        \textit{интеллекта} тех многоагентных систем, в состав которых эта
        \textit{кибернетическая система} входит}
    \scnidtf{уровень социализации кибернетической системы}
    \scnidtf{социализация}
    \scntext{примечание}{Уровень \textit{интеллекта} коллектива кибернетических систем
        (\textit{многоагентной системы}) может быть значительно ниже уровня
        \textit{интеллекта} самого глупого  члена этого коллектива, но может быть и
        значительно выше уровня \textit{интеллекта} самого умного  члена указанного
        коллектива. Для того, чтобы количество \textit{интеллектуальных систем}
        переходило в существенно более интеллектуальное качество коллектива таких
        систем, все объединяемые в коллектив \textit{интеллектуальные системы} должны
        иметь высокий уровень \textit{интероперабельности}, что накладывает
        \uline{дополнительные требования}, предъявляемые к \textit{информации, хранимой
            в памяти}, а также к \textit{решателям задач}
        %\bigspace
        \textit{интеллектуальных систем}, объединяемых в коллектив.}
    \scntext{примечание}{Коллектив \textit{кибернетических систем} может иметь
        значительно более высокий уровень качества, в том числе, уровень интеллекта,
        чем уровень качества \textit{кибернетических систем}, являющихся членами этого
        коллектива. Но так бывает не всегда. Для того, чтобы количество членов
        коллектива \textit{кибернетической системы} перешло в более высокое качество
        самого коллектива, члены коллектива должны обладать дополнительными
        способностями, которые будем называть свойствами \textit{социализации}.
        Основными такими свойствами являются способность устанавливать и поддерживать
        достаточный уровень \textit{семантической совместимости} (взаимопонимания) с
        другими кибернетическими системами и \textit{договороспособность} (способность
        согласовывать свои действия с другими).}
        \begin{scnindent}
            \scntext{источник}{\scncite{Neiva2016}}
        \end{scnindent}
    \scntext{примечание}{Целенаправленный обмен
        информацией между \textit{кибернетическими системами} существенно ускоряет
        процесс их обучения (процесс накопления знаний и навыков). Следовательно,
        способность эффективно использовать указанный канал накопления знаний и навыков
        существенно повышает уровень \textit{обучаемости}
        %\bigspace
        \textit{кибернетических систем}. В этом смысле можно сказать, что
        познавательный процесс социален.}\scnidtf{уровень развития социально значимых
        качеств кибернетической системы}
    \scntext{примечание}{Повышение уровня \textit{интероперабельности}
        %\bigspace
        \textit{кибернетической системы} является, с одной стороны, дополнительным
        повышением уровня \textit{интеллекта} самой этой \textit{кибернетической
            системы}, а также фактором повышения уровня \textit{интеллекта} тех
        коллективов, тех \textit{многоагентных систем}, в состав которых эта
        \textit{кибернетическая система} входит.}\scntext{примечание}{Переход к
        \textit{многоагентным системам} не только является важным фактором повышения
        качества \textit{кибернетических систем}, но также имеет и обратную сторону
        медали	-- появление целого ряда угроз, связанного с возможными
        целенаправленными вредоносными воздействиями на \textit{многоагентную систему}
        (со стороны некоторых ее \textit{агентов}), существенно снижающими уровень ее
        качества. Наличие таких \textit{вредоносных целей} у соответствующих
        \textit{агентов} свидетельствует о нижайшем уровне \textit{социализации} этих
        \textit{агентов}.}\scnidtf{умение согласовывать (синхронизировать) свою
        деятельность с деятельностью других кибернетических систем в процессе решения
        задач, требующих коллективных усилий}
    \scnidtf{умение участвовать в децентрализованном процессе распределения
        подзадач некоторой коллективно (распределенно) решаемой задачи между членами
        заданного коллектива кибернетических систем и умение участвовать в управлении
        коллективного решения указанной задачи}
    \begin{scnindent}
        \scntext{примечание}{Речь идет о децентрализованном асинхронном управлении деятельностью коллектива кибернетических систем}
    \end{scnindent}
    \scnidtf{способность и готовность кибернетической системы к координации своей деятельности в рамках
        коллектива кибернетических систем, в состав которого она входит в целях:
        \begin{scnitemize}

            \item эффективного решения тактических задач, решаемых указанным коллективом;
            \item решения главной стратегической задачи этого коллектива --- обеспечения как
            можно более высокой скорости роста уровня интеллекта указанного коллектива.
        \end{scnitemize}
    }
    \scntext{примечание}{Подчеркнем, что повышение уровня интеллекта коллектива
        кибернетической системы (многоагентной системы) имеет свои особенности:
        \begin{scnitemize}

            \item во-первых, это забота о семантической совместимости кибернетических
            систем входящих в состав коллектива;
            \item во-вторых, это переход от виртуальной распределенной базы знаний
            коллектива к реально поддерживаемым базам знаний и к порталам корпоративных
            знаний, реализованных в виде индивидуальных кибернетических систем, через
            которые осуществляются все процессы координации и согласования деятельности
            соответствующих членов коллектива
        \end{scnitemize}
    }
    \begin{scnrelfromlist}{свойство-предпосылка}

        \scnitem{договороспособность кибернетической системы}
        \begin{scnindent}
            \scnidtftext{часто используемый sc-идентификатор}{договороспособность}
        \end{scnindent}
        \scnitem{социальная ответственность кибернетической системы}
            \begin{scnindent}
                \scnidtftext{часто используемый sc-идентификатор}{социальная ответственность}
            \end{scnindent}
        \scnitem{социальная активность кибернетической системы}
            \begin{scnindent}
                \scnidtftext{часто используемый sc-идентификатор}{социальная активность}
            \end{scnindent}

    \end{scnrelfromlist}
    \bigskip\scnheader{договороспособность кибернетической системы}

    \begin{scnrelfromlist}{свойство-предпосылка}

        \scnitem{способность кибернетической системы к пониманию принимаемых сообщений}
        \scnitem{способность кибернетической системы к формированию передаваемых
            сообщений, понятных адресатам}
        \scnitem{семантическая совместимость кибернетической системы с партнёрами }
        \scnitem{способность кибернетической системы к обеспечению семантической
            совместимости с партнёрами }
        \scnitem{коммуникабельность кибернетической системы }
        \scnitem{способность кибернетической системы к обсуждению и согласованию целей
            и планов коллективной деятельности }
        \scnitem{способность кибернетической системы брать на себя выполнение
            актуальных задач в рамках согласованных планов коллективной деятельности}

    \end{scnrelfromlist}

    \scnheader{способность кибернетической системы к пониманию принимаемых
        сообщений}
    \scnidtf{способность кибернетической системы к пониманию информации,
        поступающей извне от других кибернетических систем}
    \scnidtf{способность кибернетической системы к отображению принимаемых
        сообщений в семантически эквивалентные фрагменты собственной базы знаний}

    \begin{scnrelfromset}{комплекс частных свойств}

        \scnitem{способность кибернетической системы к пониманию принимаемых вербальных
            сообщений }
        \scnitem{способность кибернетической системы к пониманию принимаемых
            невербальных сообщений}   

    \end{scnrelfromset}
    \scnrelfrom{свойство-предпосылка}{способность кибернетической системы к
        обеспечению семантической совместимости с партнёрами}
    
    \scnheader{сообщение}
    \scnidtf{информация, передаваемая (пересылаемая) от одной кибернетической
        системы к другой или к другим кибернетическим системам}
    \scntext{примечание}{Каждому \textit{сообщению} ставится в соответствие одна
        \textit{кибернетическая система}, являющаяся \textbf{\textit{источником
                сообщения*}} и одна или несколько \textit{кибернетических систем}, являющихся
        \textbf{\textit{адресатами сообщения*}}. В соответствии с этим для каждой
        \textit{кибернетической системы} те сообщения, \textit{источником*} которых она
        является, будем называть \textbf{\textit{передаваемыми сообщениями*}}, а те
        сообщения, \textit{адресатами*} которых она является, будем называть
        \textbf{\textit{принимаемыми сообщениями*}}.}
    \begin{scnsubdividing}

        \scnitem{вербальное сообщение}
        \begin{scnindent}
            \scnidtf{передаваемая словесная информация}
        \end{scnindent}
        \scnitem{невербальное сообщение}
        \begin{scnindent}    
            \scntext{примечание}{Примерами невербальных сообщений являются пересылаемые фото-документы, видео-материалы}
        \end{scnindent}

    \end{scnsubdividing}

    \begin{scnrelfromset}{обобщённая декомпозиция}

        \scnitem{спецификация сообщения}
            \begin{scnindent}
                \begin{scnrelfromset}{обобщённая декомпозиция}

                    \scnitem{указание источника специфицируемого сообщения}
                    \scnitem{указание множества адресатов специфицируемого сообщения}
                    \scnitem{отметка момента времени отправления специфицируемого сообщения}
                    \scnitem{указание прагматического типа специфицируемого сообщения}
                    \scnitem{указания запроса, ответом на который является специфицируемое сообщение}
                        \begin{scnindent}
                            \scntext{примечание}{Если специфицируемое сообщение является ответом на некоторый запрос}
                        \end{scnindent}
                    \scnitem{указание раздела баз знаний адресатов, которому соответствует специфицируемое сообщение}
                    \scnitem{указание способа представления тела сообщения}
                        \begin{scnindent}   
                            \scntext{примечание}{Для вербальных сообщений это указание используемого  внешнего языка}
                        \end{scnindent}
                \end{scnrelfromset}
            \end{scnindent}
        \scnitem{тело сообщения}
            \begin{scnindent}    
                \scnidtf{собственно само сообщение}
            \end{scnindent}

    \end{scnrelfromset}
    \scnrelfrom{разбиение}{прагматический тип сообщения}

    \begin{scneqtoset}

        \scnitem{повествовательное сообщение}
            \begin{scnindent}
                \scnsuperset{ответ на запрос}
            \end{scnindent}
        \scnitem{вопросительное сообщение}
        \scnitem{команда редактирования баз знаний адресатов}
        \scnitem{команда, инициирующая действие адресатов в их внешней среде}

    \end{scneqtoset}

    \scnheader{следует отличать*}
    \begin{scnhaselementset}
        \scnitem{вербальная информация}
        \scnitem{файл, содержащий вербальную информацию}
            \begin{scnindent}
                \scnidtf{вербальная информация, представленная в виде файла}
            \end{scnindent}
        \scnitem{вербальное сообщение}
    \end{scnhaselementset}

    \scnheader{вербальная информация}
    \scnidtf{знаковая конструкция, которая имеет в общем случае произвольную
        денотационную семантику и которая может либо поступать на вход кибернетической
        системы через соответствующие ее сенсоры (рецепторы), либо через
        соответствующие эффекторы передаваться (пересылаться) в качестве сообщения
        другим кибернетическим системам}

    \scnheader{следует отличать*}
    \begin{scnhaselementset}
        \scnitem{вербальная информация}
        \scnitem{сенсорная информация}
    \end{scnhaselementset}
    \begin{scnindent}
        \scntext{примечание}{И \textit{вербальная информация} и \textit{сенсорная информация}
            являются \textit{знаковыми конструкциями}, но, во-первых, \textit{вербальная
            информация} может быть как внешней знаковой конструкцией, так и внутренней
            знаковой конструкцией, хранимой в памяти кибернетической системы, а
            \textit{сенсорная информация} всегда является внутренней \textit{знаковой
            конструкцией} кибернетической системы и, во-вторых, \textit{сенсорная
            информация} описывает только пограничную  для \textit{кибернетической системы}
            физическую \textit{окружающую среду}, тогда, как \textit{вербальная информация}
            может описывать все, что угодно.}
    \end{scnindent}

    \scnheader{следует отличать*}
    \begin{scnhaselementset}
        \scnitem{невербальная информация}
        \scnitem{файл, содержащий невербальную информацию}
            \begin{scnindent}    
                \scnidtf{файл, содержимым которого является электронный образ некоторой невербальной информации}
            \end{scnindent}
        \scnitem{невербальное сообщение}
            \begin{scnindent}
            \scnidtf{невербальная информация, представленная в виде файла и передаваемая
                (пересылаемая) от одной кибернетической системы к другой}
            \end{scnindent}
        \scnitem{сенсорная информация}
            \begin{scnindent}
                \scnidtf{информация, формируемая сенсорами кибернетической системы}
            \end{scnindent}
    \end{scnhaselementset}\

    \scnheader{невербальная информация}
    \scnsuperset{музыкальное произведение}
    \scnsuperset{танец}
    \scnsuperset{произведение изобразительного искусства}
    \begin{scnindent}
        \scnsuperset{живопись}
        \scnsuperset{скульптура}
        \scnsuperset{графика}
    \end{scnindent}
    \scnsuperset{статическое изображение}
    \scnsuperset{динамическое изображение}

    \scnheader{способность кибернетической системы к пониманию принимаемых
        вербальных сообщений}
    \scnidtf{способность кибернетической системы к пониманию вербальной информации,
        поступающей извне из разных источников}
    \scntext{примечание}{Понимание информации, поступающей извне, включает в себя:
        \begin{scnitemize}

            \item перевод этой информации на внутренний язык кибернетической системы;
            \item локальную верификацию вводимой информации;
            \item погружение (конвергенцию, размещение) текста, являющегося результатом
            указанного перевода в состав хранимой информации (в частности, в состав базы
            знаний)
        \end{scnitemize}
    }\scntext{примечание}{Погружение вводимой информации в состав базы знаний
        кибернетической системы сводится к выявлению и устранению противоречий,
        возникающих между погружаемым текстом и текущего состояния базы знаний. Первым
        уровнем таких противоречий являются появляющиеся при интеграции погружаемого
        текста с текущим состоянием базы знаний \textit{омонимичные знаки} и пары
        \textit{синонимичных знаков}. Омонимичные знаки появляются в результате
        ошибочного отождествления знака, входящего в состав погружаемого текста, со
        знаком, входящим в состав погружаемого текста, со знаком, входящим в состав
        текущего состояния базы знаний. Появление пар синонимичных знаков, один из
        которых входит в погружаемый текст, а второй --- в текущее состояние базы
        знаний, при погружении вводимого текста является штатным  противоречием,
        устранение которого осуществляется путем отождествления (склеивания )
        синонимичных знаков.}\scntext{примечание}{Сложность проблемы понимания вводимой
        вербальной информации заключается не только в сложности непротиворечивого
        погружения вводимой информации в текущее состояние базы знаний, но и в
        сложности трансляции этой информации с внешнего языка на внутренний язык
        кибернетической системы, т. е. в сложности генерации текста внутреннего языка,
        семантически эквивалентного вводимому тексту внешнего языка. Очевидно, что для
        естественных языков указанная трансляция является сложной задачей, так как в
        настоящее время проблема формализации синтаксиса и семантики естественных
        языков не решена.}
        
    \scnheader{семантическая совместимость кибернетической системы с партнерами}
    \scnidtf{уровень взаимопонимания кибернетической системой со своими партнерами}
    \scnidtf{степень конвергенции (близости) базы знаний кибернетической системы с базами знаний своих партнеров}
    
    \scnheader{семантическая совместимость кибернетической системы с партнерами}
    \scnrelto{частное свойство}{\textit{совместимость кибернетических систем}}
    \scntext{пояснение}{\textit{семантическая совместимость кибернетических
            систем} определяется
        \begin{scnitemize}

            \item количеством знаков, которые хранятся в памяти одной заданной
            кибернетической системы и денотационная семантика которых совпадает с
            денотационной семантикой знаков, хранимых в памяти другой заданной
            кибернетической системы (другими словами, это количество сущностей, которые
            описывают как в памяти первой кибернетической системы, так и в памяти второй
            кибернетической системы),
            \item тем, согласованы ли между двумя заданными кибернетическими системами факт
            совпадения денотационной семантики указанных выше знаков сущностей, описываемых
            в памяти как первой, так и второй кибернетической системы (такое согласование
            осуществляется путем согласования уникальных внешних идентификаторов (имен),
            которые приписываются указанным знакам сущностей и которые используются
            указанными кибернетическими системами при обмене сообщениями между ними).
        \end{scnitemize}
    }\scntext{примечание}{Прежде всего семантическая совместимость двух заданных
        кибернетических систем определяется согласованностью систем понятий,
        используемых обеими взаимодействующими кибернетическими системами, (т.е.
        совпадением семантической трактовки всех этих понятий) и включением в число
        таких общих понятий всех или почти всех неопределяемых понятий, а также тех
        определяемых понятий, которые обеими кибернетическими системами часто
        используются при определении остальных определяемых
        понятий.}\scntext{примечание}{Высокий уровень семантической совместимости даже для
        кибернетических систем с высоким уровнем интеллекта (например, для людей)
        встречается значительно реже, чем хотелось бы. Очевидно, что проблема
        обеспечения перманентной поддержки семантической совместимости
        взаимодействующих кибернетических систем является необходимым условием
        обеспечения высокого уровня взаимопонимания кибернетических систем и, как
        следствие, эффективного их взаимодействия.}
        
    \scnheader{способность кибернетической системы к обеспечению семантической совместимости с партнерами}
    \scnidtf{способность кибернетической системы к обеспечению взаимопонимания со своими партнерами.}
    \begin{scnrelfromset}{комплекс частных свойств}
        \scnitem{способность кибернетической системы к обеспечению семантической
            совместимости собственной базы знаний с базами знаний своих партнеров}
        \scnitem{способность кибернетической системы к обеспечению коммуникационной совместимости со своими партнерами}
            \begin{scnindent}
                \scntext{примечание}{Речь идет о согласовании внешних языков, используемых кибернетическими системами при их общении.}
            \end{scnindent}
    \end{scnrelfromset}
    \begin{scnrelfromset}{комплекс частных свойств}
        \scnitem{уровень предварительной семантической совместимости кибернетической системы с партнерами}
            \begin{scnindent}
                \scntext{примечание}{Речь идет об обеспечении начальной (стартовой) семантической совместимости.}
            \end{scnindent}
        \scnitem{способность кибернетической системы к перманентной поддержке
            семантической совместимости с партнерами}
        \begin{scnindent}
            \scntext{примечание}{Речь идет о перманентном процессе поддержки
                необходимого уровня семантической совместимости(взаимопонимания) в условиях
                постоянной эволюции всех взаимодействующих кибернетических систем.}
        \end{scnindent}
    \end{scnrelfromset}

    \scnheader{уровень предварительной семантической совместимости кибернетической
        системы с партнерами}
    \scnidtf{унификация представления информации, хранимой в памяти всевозможных кибернетических систем}
    \scnidtf{максимально возможная конвергенция, стандартизация, согласованность
        представления информации, хранимой в памяти всевозможных кибернетических систем}
    \scntext{примечание}{речь идет об использовании всеми кибернетическими системами
        общего универсального языка внутреннего представления знаний и о согласовании используемых ими понятий}
        
    \scnheader{способность кибернетической системы к перманентной поддержке семантической совместимости с партнерами}
    \scnidtf{способность кибернетической системы к согласованию денотационной
        семантики знаков (и, в первую очередь, знаков понятий), используемых в
        собственной базе знаний с денотационной семантике тех знаков, которые входят в
        состав информации поступающей от других кибернетических систем-партнеров}
    \scnidtf{способность кибернетической системы к повышению уровня семантической
        совместимости и взаимопонимания с другими системами (в том числе, с
        компьютерными системами, с людьми) в условиях перманентного процесса
        собственной эволюции (следствием которой является появление новых знаковых
        понятий и других описываемых сущностей, а также уточнение денотационной
        семантики используемых знаков), перманентной эволюции партнерских
        кибернетических систем и перманентной эволюции коллективно согласованной
        картины мира}
    \scntext{примечание}{Рассматриваемое свойство (способность) кибернетической системы
        заключается в \uline{самостоятельной} реализацией перманентного (постоянного)
        процесса обеспечения поддержки своей семантической совместимости \uline{со
            всеми}(!) кибернетическими системами, с которыми данная кибернетическая система
        взаимодействует в текущий момент времени. Подчеркнем при этом, что условия
        поддержки семантической совместимости постоянно меняются --- меняется состав
        партнеров , меняются (эволюционируют) сами партнеры , эволюционирует и сама
        данная кибернетическая система}
        
    \scnheader{следует отличать*}
    \begin{scnhaselementset}
        \scnitem{cпособность кибернетической системы к обеспечению семантической
            совместимости с партнерами}
            \begin{scnindent}
                \scniselement{свойство}
                \scnrelfrom{область определения}{кибернетическая система}
            \end{scnindent}
        \scnitem{cемантическая совместимость кибернетической системы с партнерами}
            \begin{scnindent}    
                \scniselement{свойство}
                \scnrelfrom{область определения}{множество всевозможных неориентированных пар кибернетических систем*}
                    \begin{scnindent}
                        \scnidtf{множество всевозможных сочетаний кибернетических систем по две*}
                        \scnidtf{множество всевозможных двухмощных множеств кибернетических систем*}
                    \end{scnindent}
                \scnidtf{степень (уровень) семантической совместимости различных пар кибернетических систем}
            \end{scnindent}
    \end{scnhaselementset}

    \scnheader{коммуникабельность кибернетической системы}
    \scnidtftext{часто используемый sc-идентификатор}{коммуникабельность}
    \scnidtf{способность кибернетической системы к установлению взаимовыгодных
        контактов с другими кибернетическими системами (в том числе, с коллективами
        интеллектуальных систем) путем честного выявления взаимовыгодных общих целей
        (интересов).}
    \scnidtf{способность кибернетической системы к формированию новых партнерских
        связей с другими кибернетическими системами}
    
    \scnheader{способность кибернетической системы к обсуждению и согласованию
        целей и планов коллективной деятельности}
    \scnidtf{способность активно участвовать в коллективном (в согласовании
        каких-либо предложений) --- т.е. в подтверждении (признании) этих предложений,
        либо в их отклонении с указанием причин или предлагаемых доработок}
    
    \scnheader{способность кибернетической системы брать на себя выполнение
        актуальных задач в рамках согласованных планов коллективной деятельности}
    \scntext{примечание}{Данная способность кибернетической системы предполагает:
        \begin{scnitemize}
            \item учет приоритета актуальных задач;
            \item учет собственных возможностей;
            \item согласование распределения актуальных задач по исполнителям;
            \item публикацию момента начала и предполагаемого момента завершения выполнения
            указанной актуальной задачи
        \end{scnitemize}
    }
    
    \scnheader{социальная ответственность кибернетической системы}
    \begin{scnrelfromlist}{свойство-предпосылка}
        \scnitem{способность кибернетической системы выполнять качественно и в срок
            взятые на себя обязательства в рамках соответствующих коллективов}
        \scnitem{ способность кибернетической системы адекватно оценивать свои
            возможности при распределении коллективной деятельности}
        \scnitem{ альтруизм/эгоизм кибернетической системы}
        \scnitem{ отсутствие/наличие действий, которые по безграмотности
            кибернетической системы снижают качество коллективов, в состав которых она
            входит}
        \scnitem{ отсутствие/наличие осознанных , мотивированных действий, снижающих
            качество коллективов, в состав которых кибернетическая система входит}
    \end{scnrelfromlist}

    \scnheader{альтруизм/эгоизм кибернетической системы}
    \scntext{примечание}{уровень мотивации к повышеннию качества коллективов, в состав
        которых кибернетическая система входит}\scntext{эпиграф}{Надо любить науку, а
        не себя в науке.}
    \scntext{эпиграф}{Ты играешь и всем своим видом показываешь: \scnqqi{Смотрите, как я
        красиво играю}, а надо играть и показывать красоту самой музыки.}
    
    \scnheader{социальная активность кибернетической системы}
    \scnidtftext{часто используемый sc-идентификатор*}{социальная активность}
    \scnidtf{пассионарность}
    \begin{scnrelfromlist}{свойство-предпосылка}
        \scnitem{способность кибернетической системы к генерации предлагаемых целей и
            планов коллективной деятельности}
        \scnitem{активность кибернетической системы в экспертизе результатов других
            участников коллективной деятельности}
        \scnitem{способность кибернетической системы к анализу качества всех
            коллективов, в состав которых она входит, а также всех членов этих коллективов}
        \scnitem{способность кибернетической системы к участию в формировании новых
            коллективов}
        \scnitem{количество и качество тех коллективов, в состав которых
            кибернетическая система входит или входила}
    \end{scnrelfromlist}

    \scnheader{способность кибернетической системы к участию в формировании коллективов}
    \scnidtf{уровень способности в создании таких коллективов кибернетических
        систем, в состав которых входит данная кибернетическая система и которые
        направлены на коллективное решение соответствующего актуального класса сложных
        комплексных задач, с каждой из которых не может справиться любая из имеющихся
        кибернетических систем.}
    \scntext{примечание}{Формирование специализированного коллектива кибернетических
        систем сводится к тому, что в памяти каждой кибернетической системы, входящей в
        коллектив, генерируется спецификация этого коллектива, включающая в себя:
        \begin{scnitemize}
            \item перечень весь членов коллектива;
            \item способности (возможности) каждого из них;
            \item обязанности в рамках коллектива;
            \item спецификацию всего множества задач (вида деятельности), для решения
            (выполнения) которых сформирован данный коллектив кибернетических систем
        \end{scnitemize}
    }\scntext{примечание}{Каждая кибернетическая система может входить в состав большого
        количества коллективов, выполняя при этом в разных коллективах в общем случае
        разные должностные обязанности , разные
        бизнес-процессы...}\scntext{примечание}{Рассмотренный принцип формирования
        специализированного коллектива, состоящего из компьютерных систем и людей,
        фактически означает автоматизацию системной интеграции компьютерных систем и
        децентрализованный (горизонтальный) характер такой интеграции, это очевидно
        предполагает наличие достаточно высокого уровня интеллекта у интегрируемых
        компьютерных систем и людей.}
    \scnheader{количество и качество тех коллективов, в состав которых кибернетическая система входит или входила}
    \scntext{пояснение}{Данная характеристика кибернетической системы уточняет
        спектр ее социальной активности}\scntext{примечание}{Чем умнее (интеллектуальнее)
        многоагентные системы, членом которых является данная кибернетическая система,
        тем выше ее социальный статус  и перспективы быть умнее --- есть у кого
        учиться}\bigskip
\end{scnsubstruct}
\scnsourcecomment{Завершили Сегмент \scnqqi{Комплекс свойств, определяющих уровень социализации кибернетической системы как фактора существенного повышения уровня ее обучаемости, а также фактора существенного повышения качества всех тех многоагентных систем, в состав которых входит данная кибернетическая система}}
