\scseparatedfragment{Оглавление Стандарта OSTIS}
\begin{SCn}
    \scnsectionheader{Оглавление Стандарта OSTIS}
    \scnidtf{Оглавление текущей версии Стандарта OSTIS}
    \scntext{пояснение}{Иерархический перечень разделов, входящих в состав
        \textit{Стандарта OSTIS}, с дополнительной спецификацией некоторых разделов,
        указывающей альтернативные названия разделов, а также их авторов и редакторов}
    \begin{scnindent}
   	\scntext{примечание}{Существенно подчеркнуть, что иерархия разделов \textit{Стандарта
   			OSTIS} как и \textit{разделов} любой другой \textit{базы знаний} не означает
   		то, что \textit{разделы} более низкого уровня иерархии входят в состав
   		(являются частями) соответствующих разделов более высокого уровня. Связь между
   		\textit{разделами} разных уровней иерархии означает то, что \textit{раздел}
   		более низкого уровня иерархии является \textit{дочерним} разделом по отношению
   		к соответствующему \textit{разделу} более высокого уровня, т.е.
   		\textit{разделом}, который наследует свойства указанного \textit{раздела} более
   		высокого уровня.}
   	\end{scnindent}
    \scntext{примечание}{Описание логико-семантических связей каждого раздела
        \textit{Стандарта OSTIS} с другими разделами \textit{Стандарта OSTIS}
        приводится в рамках \textit{титульной спецификации} каждого \textit{раздела}.}
    \scntext{примечание}{Названия тех \textit{разделов}, которые планируется написать в
        последующих изданиях \textit{Стандарта OSTIS} или \textit{разделов}, которые
        \uline{в данном} издании \textit{Стандарта OSTIS} не печатаются, поскольку
        содержание их не изменилось по сравнению с предыдущим \uline{явно указываемым}
        изданием \textit{Стандарта OSTIS}, выделяются также \uline{жирным курсивом}, но
        для них страницы в рамках данного издания \textit{Стандарта OSTIS} не
        указываются}
    \begin{scnrelfromvector}{декомпозиция}
        \scnitem{Предисловие к первому изданию Стандарта OSTIS}
        \scnitem{Предисловие ко второму изданию Стандарта OSTIS}
        \scnitem{Оглавление Стандарта OSTIS}
        \scnitem{Титульная спецификация Стандарта OSTIS}
        \scnitem{Титульная спецификация второго издания Стандарта OSTIS}
        \scnitem{Часть 1 Стандарта OSTIS. Введение в интеллектуальные компьютерные системы нового поколения}
            \begin{scnindent}
                \begin{scnrelfromvector}{декомпозиция}
                    \scnitem{Предметная область и онтология кибернетических систем}
                        \begin{scnindent}
                            \begin{scnrelfromvector}{декомпозиция}
                                \scnitem{Предметная область и онтология многоагентных систем}
                                \scnitem{Предметная область и онтология компьютерных систем}
                            \end{scnrelfromvector}
                        \end{scnindent}
                    \scnitem{Предметная область и онтология интеллектуальных компьютерных систем нового поколения}
                    \begin{scnindent}
                        \begin{scnrelfromvector}{декомпозиция}
                            \scnitem{Предметная область и онтология смыслового представления информации}
                            \scnitem{Предметная область и онтология многоагентных моделей решателей задач, основанных на смысловом представлении информации}
                            \scnitem{Предметная область и онтология онтологических моделей интерфейсов интеллектуальных компьютерных систем, основанных на смысловом представлении информации}
                        \end{scnrelfromvector}
                    \end{scnindent}
                    \scnitem{Предметная область и онтология комплексной технологии поддержки жизненного цикла интеллектуальных компьютерных систем нового поколения}
                \end{scnrelfromvector}
            \end{scnindent}
        \scnitem{Часть 2 Стандарта OSTIS. Смысловое представление и онтологическая систематизация знаний в интеллектуальных компьютерных системах нового поколения}
            \begin{scnindent}
                \begin{scnrelfromvector}{декомпозиция}
                    \scnitem{Предметная область и онтология информационных конструкций и языков}
                    \begin{scnindent}
                        \begin{scnrelfromvector}{декомпозиция}
                            \scnitem{Предметная область и онтология внутреннего языка ostis-систем}
                                \begin{scnindent}
                                    \begin{scnrelfromvector}{декомпозиция}
                                        \scnitem{Предметная область и онтология синтаксиса внутреннего языка ostis-систем}
                                        \scnitem{Предметная область и онтология базовой денотационной семантики внутреннего языка ostis-систем}
                                    \end{scnrelfromvector}
                                \end{scnindent}
                            \scnitem{Предметная область и онтология внешних идентификаторов знаков, входящих в информационные конструкции внутреннего языка ostis-систем}
                            \scnitem{Предметная область и онтология языка внешнего графического представления информационных конструкций внутреннего языка ostis-систем}
                                \begin{scnindent}
                                    \begin{scnrelfromvector}{декомпозиция}
                                        \scnitem{Предметная область и онтология синтаксиса языка внешнего графического представления информационных конструкций внутреннего языка ostis-систем}
                                        \scnitem{Предметная область и онтология денотационной семантики языка внешнего графического представления информационных конструкций внутреннего языка ostis-систем}
                                        \scnitem{Предметная область и онтология иерархического семейства подъязыков семантически эквивалентных языку внешнего графического представления}
                                    \end{scnrelfromvector}
                                \end{scnindent}
                            \scnitem{Предметная область и онтология языка внешнего линейного представления информационных конструкций внутреннего языка ostis-систем}
                                \begin{scnindent}
                                    \begin{scnrelfromvector}{декомпозиция}
                                        \scnitem{Предметная область и онтология синтаксиса языка внешнего линейного представления информационных конструкций внутреннего языка ostis-систем}
                                        \scnitem{Предметная область и онтология денотационной семантики языка внешнего линейного представления информационных конструкций внутреннего языка ostis-систем}
                                        \scnitem{Предметная область и онтология иерархического семейства подъязыков, семантически эквивалентных языку внешнего линейного представления информационных конструкций внутреннего языка ostis-систем}
                                    \end{scnrelfromvector}
                                \end{scnindent}
                            \scnitem{Предметная область и онтология языка внешнего форматированного представления информационных конструкций внутреннего языка ostis-систем}
                                \begin{scnindent}
                                    \begin{scnrelfromvector}{декомпозиция}
                                        \scnitem{Предметная область и онтология синтаксиса языка внешнего форматированного представления информационных конструкций внутреннего языка ostis-систем}
                                        \scnitem{Предметная область и онтология денотационной семантики языка внешнего форматированного представления информационных конструкций внутреннего языка ostis-систем}
                                        \scnitem{Предметная область и онтология иерархического семейства подъязыков, семантически эквивалентных языку внешнего форматированного представления информационных конструкций внутреннего языка ostis-систем}
                                    \end{scnrelfromvector}
                                \end{scnindent}
                        \end{scnrelfromvector}
                    \end{scnindent}
                    \scnitem{Предметная область и онтология знаний и баз знаний ostis-систем}
                    \begin{scnindent}
                        \begin{scnrelfromvector}{декомпозиция}
                            \scnitem{Предметная область и онтология множеств}
                            \scnitem{Предметная область и онтология связок и отношений}
                            \scnitem{Предметная область и онтология параметров, величин и шкал}
                            \scnitem{Предметная область и онтология чисел и числовых структур}
                            \scnitem{Предметная область и онтология структур}
                            \scnitem{Предметная область и онтология темпоральных сущностей}
                                \begin{scnindent}
                                    \begin{scnrelfromvector}{декомпозиция}
                                        \scnitem{Предметная область и онтология ситуаций и событий, описывающих динамику баз знаний ostis-систем}
                                    \end{scnrelfromvector}
                                \end{scnindent}
                            \scnitem{Предметная область и онтология пространственных сущностей различных форм}
                            \scnitem{Предметная область и онтология материальных сущностей}
                            \scnitem{Предметная область и онтология семантических окрестностей}
                            \scnitem{Предметная область и онтология предметных областей}
                            \scnitem{Предметная область и онтология онтологий}
                            \scnitem{Предметная область и онтология логических формул, высказываний и логических sc-языков}
                            \scnitem{Предметная область и онтология файлов, внешних информационных конструкций и внешних языков ostis-систем}
                            \begin{scnindent}
                                \begin{scnrelfromvector}{декомпозиция}
                                    \scnitem{Предметная область и онтология естественных языков}
                                        \begin{scnindent}
                                            \begin{scnrelfromvector}{декомпозиция}
                                                \scnitem{Предметная область и онтология синтаксиса естественных языков}
                                                \scnitem{Предметная область и онтология денотационной семантики естественных языков}
                                            \end{scnrelfromvector}
                                        \end{scnindent}
                                \end{scnrelfromvector}
                            \end{scnindent}
                            \scnitem{Глобальная предметная область и онтология, описывающая воздействия, действия, методы, средства и технологии}
                                \begin{scnindent}
                                    \begin{scnrelfromvector}{декомпозиция}
                                        \scnitem{Предметная область и онтология локальных предметных областей и онтологий действий}
                                        \scnitem{Предметная область и онтология действий по управлению деятельностью многоагентных систем}
                                    \end{scnrelfromvector}
                                \end{scnindent}
                        \end{scnrelfromvector}
                    \end{scnindent}
                \end{scnrelfromvector}
            \end{scnindent}
        \scnitem{Часть 3 Стандарта OSTIS. Многоагентные решатели задач интеллектуальных компьютерных систем нового поколения}
            \begin{scnindent}
                \begin{scnrelfromvector}{декомпозиция}
                    \scnitem{Предметная область и онтология решателей задач ostis-систем}
                        \begin{scnindent}
                            \begin{scnrelfromvector}{декомпозиция}
                                \scnitem{Предметная область и онтология действий, задач, планов, протоколов и методов, реализуемых ostis-системой, а также внутренних агентов, выполняющих эти действия}
                                \scnitem{Предметная область и онтология Базового языка программирования ostis-систем}
                                    \begin{scnindent}
                                        \begin{scnrelfromvector}{декомпозиция}
                                            \scnitem{Предметная область и онтология синтаксиса Базового языка программирования ostis-систем}
                                            \scnitem{Предметная область и онтология денотационной семантики Базового языка программирования ostis-систем}
                                            \scnitem{Предметная область и онтология операционной семантики Базового языка программирования ostis-систем}
                                        \end{scnrelfromvector}
                                    \end{scnindent}
                                \scnitem{Предметная область и онтология программ и языков программирования для ostis-систем}
                                    \begin{scnindent}
                                        \begin{scnrelfromvector}{декомпозиция}
                                            \scnitem{Предметная область и онтология интерпретации современных языков программирования в ostis-системах}
                                        \end{scnrelfromvector}
                                    \end{scnindent}
                                \scnitem{Предметная область и онтология sc-языка вопросов}
                                    \begin{scnindent}
                                        \begin{scnrelfromvector}{декомпозиция}
                                            \scnitem{Предметная область и онтология синтаксиса sc-языка вопросов}
                                            \scnitem{Предметная область и онтология денотационной семантики sc-языка вопросов}
                                            \scnitem{Предметная область и онтология операционной семантики sc-языка вопросов}
                                        \end{scnrelfromvector}
                                    \end{scnindent}
                                \scnitem{Предметная область и онтология операционной семантики логических sc-языков}
                                \scnitem{Предметная область и онтология sc-языков продукционного программирования}
                                    \begin{scnindent}
                                        \begin{scnrelfromvector}{декомпозиция}
                                            \scnitem{Предметная область и онтология синтаксиса sc-языков продукционного программирования}
                                            \scnitem{Предметная область и онтология денотационной семантики sc-языков продукционного программирования}
                                            \scnitem{Предметная область и онтология операционной семантики sc-языков продукционного программирования}
                                        \end{scnrelfromvector}
                                    \end{scnindent}
                                \scnitem{Предметная область и онтология sc-моделей искусственных нейронных сетей}
                                    \begin{scnindent}
                                        \begin{scnrelfromvector}{декомпозиция}
                                            \scnitem{Предметная область и онтология синтаксиса sc-моделей искусственных нейронных сетей}
                                            \scnitem{Предметная область и онтология денотационной семантики sc-моделей искусственных нейронных сетей}
                                            \scnitem{Предметная область и онтология операционной семантики sc-моделей искусственных нейронных сетей}
                                        \end{scnrelfromvector}
                                    \end{scnindent}
                            \end{scnrelfromvector}
                        \end{scnindent}
                \end{scnrelfromvector}
            \end{scnindent}
        \scnitem{Часть 4 Стандарта OSTIS. Онтологические модели интерфейсов интеллектуальных компьютерных систем нового поколения}
        \begin{scnindent}
            \begin{scnrelfromvector}{декомпозиция}
                \scnitem{Предметная область и онтология интерфейсов ostis-систем}
                \scnitem{Предметная область и онтология интерфейсных действий пользователей ostis-системы}
                \scnitem{Предметная область и онтология сообщений, входящих в ostis-систему и выходящих из неё}
                \scnitem{Предметная область и онтология действий и внутренних агентов пользовательского интерфейса ostis-системы}
                \scnitem{Предметная область и онтология естественно-языковых интерфейсов ostis-систем}
                    \begin{scnindent}
                        \begin{scnrelfromvector}{декомпозиция}
                            \scnitem{Предметная область и онтология синтаксического анализа естественно-языковых сообщений, входящих в ostis-систему}
                            \scnitem{Предметная область и онтология понимания естественно-языковых сообщений, входящих в ostis-систему}
                            \scnitem{Предметная область и онтология синтеза естественно-языковых сообщений ostis-системы}
                        \end{scnrelfromvector}
                    \end{scnindent}
            \end{scnrelfromvector}
        \end{scnindent}
        \scnitem{Часть 5 Стандарта OSTIS. Методы и средства проектирования интеллектуальных компьютерных систем нового поколения}
            \begin{scnindent}
                \begin{scnrelfromvector}{декомпозиция}
                    \scnitem{Предметная область и онтология комплексной библиотеки многократно используемых семантически совместимых компонентов ostis-систем}
                        \begin{scnindent}
                            \begin{scnrelfromvector}{декомпозиция}
                                \scnitem{Предметная область и онтология многократно используемых компонентов баз знаний ostis-систем}
                                \scnitem{Предметная область и онтология многократно используемых компонентов решателей задач ostis-систем}
                                    \begin{scnindent}
                                        \begin{scnrelfromvector}{декомпозиция}
                                            \scnitem{Предметная область и онтология многократно используемых методов, хранимых в памяти ostis-систем и интерпретируемых их внутренними агентами}
                                            \scnitem{Предметная область и онтология многократно используемых внутренних агентов ostis-систем}
                                        \end{scnrelfromvector}
                                    \end{scnindent}
                                \scnitem{Предметная область и онтология многократно используемых компонентов интерфейсов ostis-систем}
                                \scnitem{Предметная область и онтология многократно используемых встраиваемых ostis-систем}
                            \end{scnrelfromvector}
                        \end{scnindent}
                    \scnitem{Предметная область и онтология действий и методик проектирования ostis-систем}
                        \begin{scnindent}
                            \begin{scnrelfromvector}{декомпозиция}
                                \scnitem{Предметная область и онтология действий и методик проектирования баз знаний ostis-систем}
                                \scnitem{Предметная область и онтология действий и методик проектирования решателей задач ostis-систем}
                                \scnitem{Предметная область и онтология действий и методик проектирования интерфейсов ostis-систем}
                            \end{scnrelfromvector}
                        \end{scnindent}
                    \scnitem{Предметная область и онтология действий и методик обучения проектированию ostis-систем}
                    \scnitem{Предметная область и онтология средств проектирования ostis-систем}
                        \begin{scnindent}
                            \begin{scnrelfromvector}{декомпозиция}
                                \scnitem{Логико-семантическая модель комплекса встраиваемых ostis-систем автоматизации проектирования баз знаний ostis-систем}
                                    \begin{scnindent}
                                        \begin{scnrelfromvector}{декомпозиция}
                                            \scnitem{Логико-семантическая модель ostis-системы редактирования, сборки и ввода исходных текстов различных компонентов проектируемой базы знаний в память ostis-системы}
                                            \scnitem{Логико-семантическая модель ostis-системы редактирования проектируемой базы знаний ostis-системы на уровне её внутреннего представления}
                                            \scnitem{Логико-семантическая модель ostis-системы обнаружения и анализа ошибок и противоречий в базе знаний ostis-системы}
                                            \scnitem{Логико-семантическая модель ostis-системы обнаружения и анализа информационных дыр в базе знаний ostis-системы}
                                            \scnitem{Логико-семантическая модель ostis-системы автоматизации управления взаимодействием разработчиков различных категорий в процессе проектирования базы знаний ostis-системы}
                                        \end{scnrelfromvector}
                                    \end{scnindent}
                                \scnitem{Логико-семантическая модель комплекса ostis-систем автоматизации проектирования решателей задач ostis-систем}
                                    \begin{scnindent}
                                        \begin{scnrelfromvector}{декомпозиция}
                                            \scnitem{Логико-семантическая модель ostis-системы автоматизации проектирования программ Базового языка программирования ostis-систем}
                                            \scnitem{Логико-семантическая модель ostis-системы автоматизации проектирования внутренних агентов ostis-систем, а также коллективов таких агентов}
                                            \scnitem{Логико-семантическая модель ostis-системы автоматизации проектирования искусственных нейронных сетей, семантически совместимых с базам знаний ostis-систем}
                                        \end{scnrelfromvector}
                                    \end{scnindent}
                                \scnitem{Логико-семантическая модель ostis-системы автоматизации проектирования интерфейсов ostis-систем}
                                \scnitem{Предметная область и онтология ostis-систем автоматизации проектирования различных классов ostis-систем}
                            \end{scnrelfromvector}
                        \end{scnindent}
                    \scnitem{Предметная область и онтология ostis-систем обучения проектированию ostis-систем и их компонентов}
                \end{scnrelfromvector}
            \end{scnindent}
        \scnitem{Часть 6 Стандарта OSTIS. Платформы реализации интеллектуальных компьютерных систем нового поколения}
        \begin{scnindent}
            \begin{scnrelfromvector}{декомпозиция}
                \scnitem{Предметная область и онтология методов и средств производства ostis-систем}
                    \begin{scnindent}
                        \begin{scnrelfromvector}{декомпозиция}
                            \scnitem{Предметная область и онтология базовых интерпретаторов логико-семантических моделей ostis-систем}
                                \begin{scnindent}
                                    \begin{scnrelfromvector}{декомпозиция}
                                        \scnitem{Предметная область и онтология ассоциативных семантических компьютеров для ostis-систем}
                                        \scnitem{Предметная область и онтология программных вариантов реализации базового интерпретатора логико-семантических моделей ostis-систем на современных компьютерах}
                                    \end{scnrelfromvector}
                                \end{scnindent}
                        \end{scnrelfromvector}
                    \end{scnindent}
            \end{scnrelfromvector}
        \end{scnindent}
        \scnitem{Часть 7 Стандарта OSTIS. Методы и средства реинжиниринга и эксплуатации интеллектуальных компьютерных систем нового поколения}
            \begin{scnindent}
                \begin{scnrelfromvector}{декомпозиция}
                    \scnitem{Предметная область и онтология методов и средств поддержки жизненного цикла ostis-систем}
                        \begin{scnindent}
                            \begin{scnrelfromvector}{декомпозиция}
                                \scnitem{Предметная область и онтология методов и средств реинжиниринга ostis-систем в ходе эксплуатации}
                                \scnitem{Предметная область и онтология встроенных ostis-систем поддержки использования ostis-систем конечными пользователями}
                            \end{scnrelfromvector}
                        \end{scnindent}
                \end{scnrelfromvector}
            \end{scnindent}
        \scnitem{Часть 8 Стандарта OSTIS. Экосистема интеллектуальных компьютерных систем нового поколения и их пользователей}
            \begin{scnindent}
                \begin{scnrelfromvector}{декомпозиция}
                    \scnitem{Предметная область и онтология Экосистемы OSTIS}
                    \begin{scnindent}
                        \begin{scnrelfromvector}{декомпозиция}
                            \scnitem{Предметная область и онтология автоматизируемых видов и областей человеческой деятельности}
                            \scnitem{Предметная область и онтология технологий автоматизации различных видов и областей человеческой деятельности}
                                \begin{scnindent}
                                    \begin{scnrelfromvector}{декомпозиция}
                                        \scnitem{Предметная область и онтология технологий компьютеризации различных видов и областей человеческой деятельности}
                                    \end{scnrelfromvector}
                                \end{scnindent}
                            \scnitem{Предметная область и онтология деятельности в области Искусственного интеллекта}
                            \scnitem{Логико-семантическая модель интеграции разнородных информационных ресурсов и сервисов в Экосистеме OSTIS в процессе ее расширения}
                                \begin{scnindent}
                                    \begin{scnrelfromvector}{декомпозиция}
                                        \scnitem{Предметная область и онтология библиографических источников и других информационных ресурсов}
                                    \end{scnrelfromvector}
                                \end{scnindent}
                            \scnitem{Предметная область и онтология семантически совместимых интеллектуальных ostis-порталов научных знаний}
                                \begin{scnindent}
                                    \begin{scnrelfromvector}{декомпозиция}
                                        \scnitem{Логико-семантическая модель Метасистемы OSTIS}
                                    \end{scnrelfromvector}
                                \end{scnindent}
                            \scnitem{Предметная область и онтология семантически совместимых информационно-справочных ostis-систем и интеллектуальных help-систем}
                            \scnitem{Предметная область и онтология семантически совместимых интеллектуальных корпоративных ostis-систем различного назначения}
                                \begin{scnindent}
                                    \begin{scnrelfromvector}{декомпозиция}
                                        \scnitem{Предметная область и онтология организаций}
                                    \end{scnrelfromvector}
                                \end{scnindent}
                            \scnitem{Предметная область и онтология ostis-систем, являющихся персональными ассистентами пользователей, обеспечивающими организацию эффективного взаимодействия каждого пользователя с другими ostis-системами и пользователями, входящими в состав Экосистемы OSTIS}
                                \begin{scnindent}
                                    \begin{scnrelfromvector}{декомпозиция}
                                        \scnitem{Предметная область и онтология персон}
                                    \end{scnrelfromvector}
                                \end{scnindent}
                            \scnitem{Предметная область и онтология семантически совместимых ostis-систем автоматизации образовательной деятельности}
                                \begin{scnindent}
                                    \begin{scnrelfromvector}{декомпозиция}
                                        \scnitem{Предметная область и онтология дидактических знаний}
                                    \end{scnrelfromvector}
                                \end{scnindent}
                            \scnitem{Предметная область и онтология семантически совместимых ostis-систем автоматизации проектирования и управления проектированием различных объектов}
                            \scnitem{Предметная область и онтология семантически совместимых ostis-систем автоматизации производственной деятельности}
                                \begin{scnindent}
                                    \begin{scnrelfromvector}{декомпозиция}
                                        \scnitem{Предметная область и онтология семантически совместимых ostis-систем управления рецептурным производством}
                                    \end{scnrelfromvector}
                                \end{scnindent}
                            \scnitem{Предметная область и онтология геоинформационных ostis-систем}
                                \begin{scnindent}
                                    \begin{scnrelfromvector}{декомпозиция}
                                        \scnitem{Предметная область и онтология географических объектов}
                                    \end{scnrelfromvector}
                                \end{scnindent}
                            \scnitem{Предметная область и онтология средств обеспечения информационной безопасности ostis-систем в рамках Экосистемы OSTIS}
                        \end{scnrelfromvector}
                    \end{scnindent}
                \end{scnrelfromvector}
            \end{scnindent}
        \scnitem{Библиография OSTIS}
    \end{scnrelfromvector}
    
    \scnsourcecommentinline{Завершили \textit{Оглавление
            Стандарта OSTIS}}
\end{SCn}