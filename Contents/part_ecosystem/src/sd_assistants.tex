\begin{SCn}
    \scnsectionheader{Предметная область и онтология ostis-систем, являющихся персональными ассистентами пользователей, обеспечивающими организацию эффективного взаимодействия каждого пользователя с другими ostis-системами и пользователями, входящими в состав Экосистемы OSTIS}
    \begin{scnsubstruct}
        \begin{scnrelfromlist}{библиографическая ссылка}
            \scnitem{\scncite{Meurisch2017}}
            \scnitem{\scncite{Meurisch2020}}
            \scnitem{\scncite{Jeni2022}}
            \scnitem{\scncite{Akbar2022}}
        \end{scnrelfromlist}

        \scnheader{Предметная область ostis-систем, являющихся персональными ассистентами пользователей в рамках Экосистемы OSTIS}
        \scniselement{предметная область}
        \begin{scnhaselementrole}{максимальный класс объектов исследования}
            {персональный ostis-ассистент}
        \end{scnhaselementrole}
        \begin{scnhaselementrolelist}{класс объектов исследования}
            \scnitem{персональный ассистент}
        \end{scnhaselementrolelist}

        \scnheader{персональный ассистент}
        \begin{scnrelfromlist}{пояснение}
            \scnfileitem{Общество должно обеспечивать персональную поддержку каждому человеку, учитывая его индивидуальные особенности, с целью достижения следующих целей:
                \begin{itemize}
                    \item максимального уровня физического здоровья, активности и долголетия;
                    \item максимального уровня физического комфорта, личного пространства и материального благосостояния;
                    \item максимального уровня социального комфорта и защиты прав и свобод.
                \end{itemize}}
            \scnfileitem{Должен осуществляться:
                \begin{itemize}
                    \item персональный мониторинг каждой личности по всем направлениям;
                    \item диагностика и устранение нежелательных отклонений;
                    \item оказание своевременной персональной помощи в уточнении направлений дальнейшей эволюции каждой личности.
                \end{itemize}}
            \scnfileitem{Необходимо перейти от оказания услуг в решении различных проблем по инициативе самих лиц, столкнувшихся с этими проблемами, к своевременному обнаружению возможности возникновения этих проблем и к соответствующей профилактике. Это возможно только при наличии четкой системной организации персонального мониторинга.}
            \scnfileitem{Цифровые \textit{персональные ассистенты} --- это программы, основанные на технологиях искусственного интеллекта и машинного обучения, которые помогают пользователям в выполнении повседневных задач, таких как составление расписания, управление контактами, поиск информации, напоминание о важных событиях и так далее.}
            \begin{scnindent}
                \begin{scnrelfromset}{смотрите}
                    \scnitem{\scncite{Meurisch2017}}
                    \scnitem{\scncite{Meurisch2020}}
                    \scnitem{\scncite{Jeni2022}}
                    \scnitem{\scncite{Akbar2022}}
                \end{scnrelfromset}
            \end{scnindent}
            \scnfileitem{\textit{Персональный ассистент} должен учитывать, что роли пользователя в обществе могут меняться, расширяться, также как и его интересы и цели. При этом, все \textit{персональные ассистенты} должны быть семантически совместимыми с целью понимания друг друга, а также обладать способностью самостоятельно взаимодействовать в рамках различных \textit{корпоративных систем}, представляя интересы своих пользователей.}
            \scnfileitem{Одной из основных проблем, связанных с реализацией цифровых \textit{персональных ассистентов}, является необходимость точного понимания запросов и задач, поступающих от пользователя. Это может быть вызвано различными факторами, такими как нечеткость и неоднозначность формулировок, использование аббревиатур и сленга, а также многозначность некоторых слов.}
            \scnfileitem{Пользователь не обязан знать множество сервисов, из которых он должен выбирать подходящий ему функционал. Комплекс семантически совместимых сервисов должен располагаться "за кадром"{}. Следовательно, все используемые информационные ресурсы и сервисы должны быть семантически совместимы. Выбор подходящего для пользователя ресурса или сервиса должен производить его \textit{персональный ассистент}.}
            \scnfileitem{При реализации цифровых \textit{персональных ассистентов} необходимо обеспечить их масштабируемость и адаптивность к потребностям пользователей. Это означает, что система должна быть способна автоматически адаптироваться к изменениям в поведении пользователя, учитывая его предпочтения, особенности работы и другие факторы.}
        \end{scnrelfromlist}

        \scnheader{персональный ostis-ассистент}
        \scnidtf{ostis-система, являющаяся персональным ассистентом пользователя в рамках Экосистемы OSTIS}
        \begin{scnrelfromset}{возможности}
            \scnfileitem{Возможность анализа деятельности пользователя и формирования рекомендаций по ее оптимизации.}
            \scnfileitem{Возможность адаптации под настроение пользователя, его личностные качества, общую окружающую обстановку, задачи, которые чаще всего решает пользователь.}
            \scnfileitem{Перманентное обучение самого ассистента в процессе решения новых задач, при этом обучаемость потенциально не ограничена.}
            \scnfileitem{Возможность вести диалог с пользователем на естественном языке, в том числе в речевой форме.}
            \scnfileitem{Возможность отвечать на вопросы различных классов, при этом если системе что-то не понятно, то она сама может задавать встречные вопросы.}
            \scnfileitem{Возможность автономного получения информации от всей окружающей среды, а не только от пользователя (в текстовой или речевой форме). При этом система может как анализировать доступные информационные источники (например, в интернете), так и анализировать окружающий ее физический мир, например, окружающие предметы или внешний вид пользователя.}
        \end{scnrelfromset}
        \begin{scnindent}
        	\scntext{примечание}{При этом система может как анализировать доступные информационные источники (например, в интернете), так и анализировать окружающий ее физический мир, например, окружающие предметы или внешний вид пользователя.}
        \end{scnindent}
        \begin{scnrelfromset}{достоинства}
            \scnfileitem{Пользователю нет необходимости хранить разную информацию в разной форме в разных местах, вся информация хранится в единой базе знаний компактно и без дублирований.}
            \scnfileitem{Благодаря неограниченной обучаемости ассистенты могут потенциально автоматизировать практически любую деятельность, а не только самую рутинную.}
            \scnfileitem{Благодаря базе знаний, ее структуризации и средствам поиска информации в базе знаний пользователь может получить более точную информацию более быстро.}
        \end{scnrelfromset}
        \scnsuperset{персональный ostis-ассистент учебного назначения}
        \begin{scnindent}
        	\scnidtf{персональный ассистент-учитель}
        \end{scnindent}
        \scnsuperset{персональный ostis-ассистент по здоровому образу жизни и здоровому питанию}
        \begin{scnindent}
        	\scnidtf{персональный фитнесс-тренер}
        \end{scnindent}
        \scnsuperset{персональный ostis-ассистент для ухода за пациентом}
        \scnsuperset{персональный секретарь-референт}
        \scntext{примечание}{\textit{Персональные ассистенты} имеют самое различное назначение и могут быть использованы для самых различных категорий пользователей (пациент, юридическое обслуживание, административное обслуживание, покупатель, потребитель различных услуг). \textit{Персональный ostis-ассистент} может использовать знания и данные, хранящиеся в других \textit{ostis-системах}, таких как \textit{корпоративные ostis-системы}, чтобы предоставлять пользователю более полную и актуальную информацию. Это может быть особенно полезно для пользователей, которые работают с большим количеством данных и информации. \textit{Персональный ostis-ассистент} автоматически интегрируется с другими \textit{ostis-системами}, что позволяет ему более эффективно работать с данными и информацией. Он может использовать технологии машинного обучения и искусственного интеллекта для адаптации к поведению пользователя и улучшения его производительности и эффективности. \textit{Персональный ostis-ассистент} может быть создан и настроен с учетом конкретных потребностей организации и ее процессов, что может привести к значительным экономическим и производственным преимуществам.}
        \scntext{примечание}{\textit{Персональные ostis-ассистенты} обладают рядом преимуществ по сравнению с другими реализациями цифровых \textit{персональных ассистентов}, таких как более точное понимание запросов и задач пользователей, доступ к актуальным данным и информации, автоматическая интеграция с другими \textit{ostis-системами} в рамках \textit{Экосистемы OSTIS} и адаптация к потребностям организации и ее процессов.}
        \scntext{примечание}{Каждой персоне, входящей в состав \textit{Экосистемы OSTIS} взаимно однозначно соответствует его личный (персональный) ассистент в виде \textit{персонального ostis-ассистента}.
        Таким образом, количество \textit{персональных ostis-ассистентов}, входящих в состав \textit{Экосистемы OSTIS}, совпадает с числом персон, входящих в состав \textit{Экосистемы OSTIS}.}
        \begin{scnindent}
            \begin{scnrelfromset}{смотрите}
                \scnitem{SCg-текст. Пример персоны и соответствующего ему персонального ostis-ассистента}
            \end{scnrelfromset}
        \end{scnindent}

        \scnheader{SCg-текст. Пример персоны и соответствующего ему персонального ostis-ассистента}
        \scnrelfrom{иллюстрация}{\scnfileimage[20em]{Contents/part_ecosystem/src/images/personal_ostis_assistant_example_ru.png}}
    
    \end{scnsubstruct}
    \scnendcurrentsectioncomment
\end{SCn}
